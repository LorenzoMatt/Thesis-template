% !TEX encoding = UTF-8
% !TEX TS-program = pdflatex
% !TEX root = ../tesi.tex

%**************************************************************
% Sommario
%**************************************************************
\cleardoublepage
\phantomsection
\pdfbookmark{Sommario}{Sommario}
\begingroup
\let\clearpage\relax
\let\cleardoublepage\relax
\let\cleardoublepage\relax

\chapter*{Sommario}

Il presente documento descrive il lavoro svolto durante il periodo di stage, della durata di 320 ore, dal laureando \myName presso l'azienda \myCompany, situata a Padova.
Lo stage ha avuto come argomento principale l'integrazione di un nuovo microservizio nel \textit{back-end} dell'applicazione \textbf{\textit{SportWill}}.
La \textit{web app} di \textbf{\textit{SportWill}} dà modo all'utente di divulgare la sua intenzione (\textit{will}) di effettuare un'attività sportiva. La piattaforma fa vedere tutto a tutti, rendendo troppo caotica la fruizione. 
L'esigenza è di dare la possibilità all'utente (con le nuove implementazioni) di creare uno o più gruppi a cui utenti \enquote*{amici} possano unirsi, quindi vedere solo le \enquote*{will} di gruppo.
Le attività svolte nel corso dello stage sono \textbf{DA SCRIVERE}.
La prima è un \gls{microservizio} Java sviluppato mediante il \gls{framework} \gls{Spring} che consente di effettuare le funzionalità \gls{CRUD} sui gruppi, 
la visualizzazione delle \enquote*{will} degli utenti appartenenti allo stesso gruppo.\\
La seconda è la modifica dei microservizi esistenti affinchè le \enquote*{Will} possano essere visualizzate o da tutti gli utenti oppure solo dagli utenti appartenenti agli stessi gruppi.\\
La terza è l'implementazione di un microservizio, sviluppato sempre mediante il \textit{framework} Spring, che permette di aggregare le chiamate 
alle \gls{api} dei vari servizi presenti e restituire i risultati appropriati.
La quarta è l'implementazione di un \gls{Eureka Server} un microservizio che contiene le informazioni su tutti i servizi che si registrano nel suo server.
La quinta è la \gls{containerizzazione} di tutti i microservizi su \gls{Docker}.
L'ultima attività è stata la modifica del \textit{front-end}, mediante il framework \textbf{Angular}, per adeguarla alle nuove funzionalità del \textit{back-end}.
%\vfill
%
%\selectlanguage{english}
%\pdfbookmark{Abstract}{Abstract}
%\chapter*{Abstract}
%
%\selectlanguage{italian}

\endgroup			

\vfill

