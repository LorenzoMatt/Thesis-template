% !TEX encoding = UTF-8
% !TEX TS-program = pdflatex
% !TEX root = ../tesi.tex

%**************************************************************
% Sommario
%**************************************************************
\cleardoublepage
\phantomsection
\pdfbookmark{Sommario}{Sommario}
\begingroup
\let\clearpage\relax
\let\cleardoublepage\relax
\let\cleardoublepage\relax

\chapter*{Sommario}

Il presente documento descrive il lavoro svolto durante il periodo di
\textit{stage},
della durata di 320 ore, dal laureando \myName presso
l'azienda \myCompany, situata a Padova.
Lo \textit{stage} ha avuto come argomento principale l'implementazione di nuove
funzionalità nel contesto dell'applicazione \productName, una
\textit{web app} che dà modo all'utente di divulgare la sua intenzione
(\textit{will}) di effettuare un'attività sportiva.
La piattaforma permetteva ad ogni utente l'accesso a tutte le informazioni,
rendendo troppo caotica la
fruizione, quindi l'esigenza era di dare la possibilità all'utente
di creare uno o più gruppi a cui utenti \enquote*{amici} potessero unirsi,
vedendo quindi solo le \textit{will} relative al gruppo.
Le attività svolte nel corso dello \textit{stage} sono state due:
\begin{itemize}
      \item l'aggiornamento del \textit{back
                  end} della \textit{web app}, mediante lo sviluppo di tre
            microservizi
            mediante il \textit{framework} Spring Java.
            In particolare:
            \begin{enumerate}
                  \item il primo microservizio	gestisce i gruppi e la
                        visualizzazione delle
                        \textit{will}
                        degli utenti appartenenti allo stesso gruppo;
                  \item il secondo è l'implementazione di un
                        \texttt{API Gateway}, che permette di esporre le
                        API dei vari servizi
                        presenti in un
                        unico punto di accesso;
                  \item il terzo è l'implementazione di un \texttt{Eureka
                              Server}
                        che
                        contiene le informazioni di tutti i servizi che si
                        registrano nel suo server.
            \end{enumerate}
            Inoltre i microservizi esistenti sono stati modificati affinché le
            \textit{will} possano essere visualizzate o da tutti gli utenti oppure solo
            dagli utenti appartenenti agli stessi gruppi.
            Ultimate le attività lato \textit{back end}, è stata effettuata la
            \textit{containerizzazione} di tutti i microservizi su
            Docker.
      \item La modifica del \textit{front end},
            mediante il \textit{framework} Angular, per adeguarlo alle nuove
            funzionalità del
            \textit{back end}.
\end{itemize}
% L'esito dello \textit{stage} è stato molto positivo: le attività obbligatorie e
% facoltative sono state portate a termine con successo abbastanza facilmente
% e con un un po' di anticipo, permettendomi così di implementare anche la parte
% \textit{front end} dell'applicazione.

%\vfill
%
%\selectlanguage{english}
%\pdfbookmark{Abstract}{Abstract}
%\chapter*{Abstract}
%
%\selectlanguage{italian}

\endgroup

\vfill