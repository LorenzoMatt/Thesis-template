% !TEX encoding = UTF-8
% !TEX TS-program = pdflatex
% !TEX root = ../tesi.tex

%**************************************************************
% Sommario
%**************************************************************
\cleardoublepage
\phantomsection
\pdfbookmark{Sommario}{Sommario}
\begingroup
\let\clearpage\relax
\let\cleardoublepage\relax
\let\cleardoublepage\relax

\chapter*{Sommario}

Il presente documento descrive il lavoro svolto durante il periodo di stage, della durata di 320 ore, dal laureando \myName\xspace presso
 l'azienda \myCompany, situata a Padova.
Lo stage ha avuto come argomento principale l'implementazione di nuove funzionalità nel contesto dell'applicazione \productName, una
\textit{web app} che dà modo all'utente di divulgare la sua intenzione (\gls{will}) di effettuare un'attività sportiva.
La piattaforma fa vedere tutto a tutti, rendendo troppo caotica la fruizione, quindi l'esigenza era di dare la possibilità all'utente 
di creare uno o più gruppi a cui utenti \enquote*{amici} possano unirsi, vedendo quindi solo le \gls{will} di gruppo.
Le attività svolte nel corso dello stage sono due:
\begin{itemize}
    \item la prima è un insieme di attività che sono legate al \textit{back-end} della \textit{web app}, come lo sviluppo di tre \glspl{microservizio} mediante il \gls{framework} \gls{Spring} Java. 
    In particolare: 
        \begin{enumerate}
            \item il primo microservizio consente l'effettuazione delle funzionalità \gls{CRUD} per la gestione dei gruppi e la visualizzazione delle \gls{will} 
            degli utenti appartenenti allo stesso gruppo;
            \item il secondo è l'implementazione di un \gls{API Gateway}, che permette di esporre le \gls{api} dei vari servizi presenti in un 
            unico punto di accesso;
            \item il terzo è l'implementazione di un \gls{Eureka Server} che contiene le informazioni di tutti i servizi che si registrano nel suo server.
        \end{enumerate}
    Oltre all'implementazione di nuovi microservizi, quelli esistenti sono stati modificati affinchè le \gls{will} possano
    essere visualizzate o da tutti gli utenti oppure solo dagli utenti appartenenti agli stessi gruppi. 
    Ultimate le attività lato \textit{back-end}, è stata effettuata la \gls{containerizzazione} di tutti i microservizi su Docker.
    \item La seconda attività è stata la modifica del \textit{front-end}, mediante il framework \gls{Angular}, per adeguarla alle nuove funzionalità del \textit{back-end}.
\end{itemize}
L'esito dello stage è stato molto positivo: le attività obbligatorie e facoltative sono state portate a termine con successo abbastanza facilmente 
e con un un po' di anticipo che mi ha permesso di implementare anche la parte \textit{front-end} dell'applicazione.

%\vfill
%
%\selectlanguage{english}
%\pdfbookmark{Abstract}{Abstract}
%\chapter*{Abstract}
%
%\selectlanguage{italian}

\endgroup			

\vfill

