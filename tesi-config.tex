%**************************************************************
% file contenente le impostazioni della tesi
%**************************************************************

%**************************************************************
% Frontespizio
%**************************************************************

% Autore

\newcommand{\mappaturaemetodo}[2]{\textbf{Mappatura}: \texttt{#1}, \textbf{verbo HTTP}: #2}
\newcommand{\class}[1]{\textit{\textbf{#1}}}
\newcommand{\myparagraph}[1]{\paragraph{#1}\mbox{}\\}
\newcommand{\mysubparagraph}[1]{\subparagraph{#1}\mbox{}\vspace{3pt} \\}
\makeatletter
\newcounter{subsubparagraph}[subparagraph]
\renewcommand\thesubsubparagraph{%
  \thesubparagraph.\@arabic\c@subsubparagraph}
\newcommand\subsubparagraph{%
  \@startsection{subsubparagraph}    % counter
    {6}                              % level
    {\parindent}                     % indent
    {3.25ex \@plus 1ex \@minus .2ex} % beforeskip
    {-1em}                           % afterskip
    {\normalfont\normalsize\bfseries}}
\newcommand\l@subsubparagraph{\@dottedtocline{6}{10em}{5em}}
\newcommand{\subsubparagraphmark}[1]{}
\makeatother

\definecolor{codegray}{gray}{0.9}
\newcommand{\code}[1]{\colorbox{codegray}{\texttt{#1}}}
\newcommand{\myName}{Lorenzo Matterazzo\xspace} 
\newcommand{\myCompany}{Sync Lab S.r.l.\xspace}                                    
\newcommand{\myTitle}{Integrazione di nuove funzionalità ad un applicativo web utilizzando Spring e Angular}
\newcommand{\productName}{\textit{\textbf{SportWill}}\xspace}
% Tipo di tesi                   
\newcommand{\myDegree}{Tesi di laurea triennale}

% Università             
\newcommand{\myUni}{Università degli Studi di Padova}

% Facoltà       
\newcommand{\myFaculty}{Corso di Laurea in Informatica}

% Dipartimento
\newcommand{\myDepartment}{Dipartimento di Matematica "Tullio Levi-Civita"}

% Titolo del relatore
\newcommand{\profTitle}{Prof.}

% Relatore
\newcommand{\myProf}{Paolo Baldan}

% Luogo
\newcommand{\myLocation}{Padova}

% Anno accademico
\newcommand{\myAA}{2020-2021}

% Data discussione
\newcommand{\myTime}{Dicembre 2021}


%**************************************************************
% Impostazioni di impaginazione
% see: http://wwwcdf.pd.infn.it/AppuntiLinux/a2547.htm
%**************************************************************

\setlength{\parindent}{14pt}   % larghezza rientro della prima riga
\setlength{\parskip}{0pt}   % distanza tra i paragrafi


%**************************************************************
% Impostazioni di biblatex
%**************************************************************
\bibliography{bibliografia} % database di biblatex 

\defbibheading{bibliography} {
    \cleardoublepage
    \phantomsection 
    \addcontentsline{toc}{chapter}{\bibname}
    \chapter*{\bibname\markboth{\bibname}{\bibname}}
}

\setlength\bibitemsep{1.5\itemsep} % spazio tra entry

\DeclareBibliographyCategory{opere}
\DeclareBibliographyCategory{web}

\addtocategory{opere}{womak:lean-thinking}
\addtocategory{web}{site:agile-manifesto}

\defbibheading{opere}{\section*{Riferimenti bibliografici}}
\defbibheading{web}{\section*{Siti Web consultati}}


%**************************************************************
% Impostazioni di caption
%**************************************************************
\captionsetup{
    tableposition=top,
    figureposition=bottom,
    font=small,
    format=hang,
    labelfont=bf
}

%**************************************************************
% Impostazioni di glossaries
%**************************************************************

%**************************************************************
% Acronimi
%**************************************************************
\renewcommand{\acronymname}{Acronimi e abbreviazioni}

\newacronym[description={\glslink{apig}{Application Program Interface}}]
    {API}{API}{Application Program Interface}

\newacronym[description={\glslink{umlg}{Unified Modeling Language}}]
    {uml}{UML}{Unified Modeling Language}

\newacronym[description={\glslink{spag}{Single-Page Application}}]{spa}{SPA}{Single-Page Application}


\newacronym[description={\glslink{IPg}{Internet Protocol address}}]{IP}{IP}{Internet Protocol address}


\newacronym[first = {URL}]{URL}{URL}{
        Uniform Resource Locator}


\newacronym[description={\glslink{CRUDg}{Create Read Update Delete}}]
    {CRUD}{CRUD}{Create Read Update Delete}

    \newacronym[description={\glslink{JSPg}{JavaServer Pages}}]
    {JSP}{JSP}{JavaServer Pages}

\newacronym[description={\glslink{ICTg}{Information and Comunications Tecnology}}]
    {ICT}{ICT}{Information and Comunications Tecnology}
%**************************************************************
% Glossario
%**************************************************************
%\renewcommand{\glossaryname}{Glossario}

\newglossaryentry{apig}
{
    name=\glslink{api}{API},
    text=Application Program Interface,
    sort=api,
    description={in informatica con il termine \emph{Application Programming Interface API} (ing. interfaccia di programmazione di un'applicazione) si indica ogni insieme di procedure disponibili al programmatore, di solito raggruppate a formare un set di strumenti specifici per l'espletamento di un determinato compito all'interno di un certo programma. La finalità è ottenere un'astrazione, di solito tra l'hardware e il programmatore o tra software a basso e quello ad alto livello semplificando così il lavoro di programmazione}
}

\newglossaryentry{superset}
{
    name=\textit{superset},
    description={Un linguaggio di programmazione che contiene tutte le funzionalità di un determinato linguaggi, però ampliandolo o migliorandolo per includere anche altre funzionalità}
}
\newglossaryentry{endpoint}
{
    name=\textit{endpoint},
    description={Canale da cui le \gls{API} possono accedere alle risorse di cui hanno bisogno per eseguire la loro funzione}
}


\newglossaryentry{JSPg}
{
    name=JSP,
    description={documento di testo, scritto con una specifica sintassi, che rappresenta una pagina \textit{web} di contenuto parzialmente o totalmente dinamico. Elaborando la pagina JSP, il motore JSP produce dinamicamente la pagina HTML finale che verrà rappresentata nel \textit{browser} dell'utente}
}

\newglossaryentry{IPg}
{
    name=IP,
    description={codice numerico usato da tutti i dispositivi (\textit{computer}, \textit{server web}, stampanti, \textit{modem}) per navigare in Internet e per comunicare in una rete locale. Un indirizzo IP costituisce quindi la base per una trasmissione corretta delle informazioni dal mittente al ricevente.}
}

\newglossaryentry{spag}
{
    name=SPA,
    description={Una \textit{single-page application} (SPA) è un'applicazione \textit{web} o un sito \textit{web} che interagisce con l'utente riscrivendo dinamicamente la pagina \textit{web} corrente con nuovi dati dal server, invece del metodo predefinito di un \textit{browser} \textit{web} che carica intere nuove pagine.}
}

\newglossaryentry{umlg}
{
    name=\glslink{uml}{UML},
    text=UML,
    sort=uml,
    description={in ingegneria del software \emph{UML, Unified Modeling Language} (ing. linguaggio di modellazione unificato) è un linguaggio di modellazione e specifica basato sul paradigma object-oriented. L'\emph{UML} svolge un'importantissima funzione di ``lingua franca'' nella comunità della progettazione e programmazione a oggetti. Gran parte della letteratura di settore usa tale linguaggio per descrivere soluzioni analitiche e progettuali in modo sintetico e comprensibile a un vasto pubblico}
}
\newglossaryentry{microservizio}
{
    name={microservizio},
    description={},
    plural = {microservizi}
}

\newglossaryentry{transcompilazione}
{
    name={transcompilazione},
    description={Tipo di traduzione 
    che prende come input il codice sorgente di un programma scritto in un linguaggio di programmazione e produce un codice sorgente equivalente nello stesso o in un linguaggio di programmazione diverso.},
}


\newglossaryentry{framework}
{
    name={framework},
    description={}
}
\newglossaryentry{Spring}
{
    name={Spring},
    description={}
}

\newglossaryentry{CRUDg}
{
    name={CRUD},
    description={Operazioni di base che possono essere svolte su un database. In particolare,
    sono creazione (Create), lettura (Read), modifica/aggiornamento (Update) ed
    eliminazione (Delete).}
}

\newglossaryentry{API Gateway}
{
    name={API Gateway},
    description={}
}


\newglossaryentry{SRP}
{
    name=\textit{Single Responsibility Principle},
    description={Nella programmazione orientata agli oggetti, il principio di singola responsabilità afferma che ogni elemento di un programma deve avere una sola responsabilità, e che tale responsabilità debba essere interamente incapsulata dall'elemento stesso.}
}

\newglossaryentry{Eureka Server}
{
    name={Eureka Server},
    description={}
}


\newglossaryentry{will}
{
    name={\textit{will}},
    description={Intenzione di un utente di fare sport.}
}


\newglossaryentry{containerizzazione}
{
    name={containerizzazione},
    description={}
}

\newglossaryentry{Angular}
{
    name={Angular},
    description={}
}

\newglossaryentry{Trello}
{
    name={Trello},
    description={}
}

\newglossaryentry{ICTg}
{
    name={ICT},
    description={in informatica con il termine \textit{Information and Comunications Tecnology} 
    si indica l'insieme dei metodi e delle tecniche utilizzate nella trasmissione, ricezione 
    ed elaborazione di dati e informazioni.}
}


% Customize the format of the first use.  See the manual for details if
% you want to include more information here such as the definition.
\defglsdisplayfirst[\glsdefaulttype]{#1\glsfirstoccur}
 % database di termini
\makeglossaries


%**************************************************************
% Impostazioni di graphicx
%**************************************************************
\graphicspath{{immagini/}} % cartella dove sono riposte le immagini


%**************************************************************
% Impostazioni di hyperref
%**************************************************************
\hypersetup{
    %hyperfootnotes=false,
    %pdfpagelabels,
    %draft,	% = elimina tutti i link (utile per stampe in bianco e nero)
    colorlinks=true,
    linktocpage=true,
    pdfstartpage=1,
    pdfstartview=,
    % decommenta la riga seguente per avere link in nero (per esempio per la stampa in bianco e nero)
    %colorlinks=false, linktocpage=false, pdfborder={0 0 0}, pdfstartpage=1, pdfstartview=FitV,
    breaklinks=true,
    pdfpagemode=UseNone,
    pageanchor=true,
    pdfpagemode=UseOutlines,
    plainpages=false,
    bookmarksnumbered,
    bookmarksopen=true,
    bookmarksopenlevel=1,
    hypertexnames=true,
    pdfhighlight=/O,
    %nesting=true,
    %frenchlinks,
    urlcolor=webbrown,
    linkcolor=RoyalBlue,
    citecolor=webgreen,
    %pagecolor=RoyalBlue,
    %urlcolor=Black, linkcolor=Black, citecolor=Black, %pagecolor=Black,
    pdftitle={\myTitle},
    pdfauthor={\textcopyright\ \myName, \myUni, \myFaculty},
    pdfsubject={},
    pdfkeywords={},
    pdfcreator={pdfLaTeX},
    pdfproducer={LaTeX}
}

%**************************************************************
% Impostazioni di itemize
%**************************************************************
\renewcommand{\labelitemi}{$\ast$}

%\renewcommand{\labelitemi}{$\bullet$}
%\renewcommand{\labelitemii}{$\cdot$}
%\renewcommand{\labelitemiii}{$\diamond$}
%\renewcommand{\labelitemiv}{$\ast$}


%**************************************************************
% Impostazioni di listings
%**************************************************************
\lstset{
    language=[LaTeX]Tex,%C++,
    keywordstyle=\color{RoyalBlue}, %\bfseries,
    basicstyle=\small\ttfamily,
    %identifierstyle=\color{NavyBlue},
    commentstyle=\color{Green}\ttfamily,
    stringstyle=\rmfamily,
    numbers=none, %left,%
    numberstyle=\scriptsize, %\tiny
    stepnumber=5,
    numbersep=8pt,
    showstringspaces=false,
    breaklines=true,
    frameround=ftff,
    frame=single
} 


%**************************************************************
% Impostazioni di xcolor
%**************************************************************
\definecolor{webgreen}{rgb}{0,.5,0}
\definecolor{webbrown}{rgb}{.6,0,0}


%**************************************************************
% Altro
%**************************************************************

\newcommand{\omissis}{[\dots\negthinspace]} % produce [...]

% eccezioni all'algoritmo di sillabazione
\hyphenation
{
    ma-cro-istru-zio-ne
    gi-ral-din
}

\newcommand{\sectionname}{sezione}
\addto\captionsitalian{\renewcommand{\figurename}{Figura}
                       \renewcommand{\tablename}{Tabella}}

\newcommand{\glsfirstoccur}{\ap{{[g]}}}

\newcommand{\intro}[1]{\emph{\textsf{#1}}}

%**************************************************************
% Environment per ``rischi''
%**************************************************************
\newcounter{riskcounter}                % define a counter
\setcounter{riskcounter}{0}             % set the counter to some initial value

%%%% Parameters
% #1: Title
\newenvironment{risk}[1]{
    \refstepcounter{riskcounter}        % increment counter
    \par \noindent                      % start new paragraph
    \textbf{\arabic{riskcounter}. #1}   % display the title before the 
                                        % content of the environment is displayed 
}{
    \par\medskip
}

\newcommand{\riskname}{Rischio}

\newcommand{\riskdescription}[1]{\textbf{\\Descrizione:} #1.}

\newcommand{\risksolution}[1]{\textbf{\\Piano di contingenza:} #1.}

%**************************************************************
% Environment per ``use case''
%**************************************************************
\newcounter{usecasecounter}             % define a counter
\setcounter{usecasecounter}{0}          % set the counter to some initial value

\makeatletter

%%%% Parameters
% #1: Nome
\newenvironment{usecase}[1]{
    \resetcounter
    \renewcommand{\theusecasecounter}{\usecasename #1}  % this is where the display of                                                         % the counter is overwritten/modified
    \refstepcounter{usecasecounter}             % increment counter
    \protected@edef\@currentlabelname{\textbf{\usecasename \arabic{usecasecounter}}}% addition here
    \vspace{10pt}
    \par \noindent                              % start new paragraph
    {\large \textbf{\usecasename \arabic{usecasecounter}: #1}}       % display the title before the 
                                                % content of the environment is displayed 
    \medskip
}{
    \medskip
}





%**************************************************************
% titlespacing settings
%**************************************************************
\titlespacing*{\section}{0pt}{1.1\baselineskip}{\baselineskip}
\titlespacing*{\subsection}{0pt}{1.1\baselineskip}{\baselineskip}
% \titlespacing*{\subparagraph}{0pt}{1.1\baselineskip}{\baselineskip}



%**************************************************************
% titleformat settings
%**************************************************************
\titleformat{\subparagraph}[runin]{\normalfont\normalsize\bfseries}{}{0pt}{}
\titleformat{\paragraph}[runin]{\normalfont\normalsize\bfseries}{}{0pt}{}


%**************************************************************
% Environment per ``sub use case''
%**************************************************************
\newcounter{subusecasecounter}             % define a counter
\setcounter{subusecasecounter}{0}          % set the counter to some initial value



\newenvironment{subusecase}[1]{
    \setcounter{subsubusecasecounter}{0}
    \refstepcounter{subusecasecounter}             % increment counter
    \protected@edef\@currentlabelname{\textbf{\usecasename \arabic{usecasecounter}.\arabic{subusecasecounter}}}% addition here

    % \def\@currentlabelname{\textbf{\usecasename \arabic{usecasecounter}}%
    \vspace{10pt}
    \noindent                              % start new paragraph
    {\large \textbf{\usecasename \arabic{usecasecounter}.\arabic{subusecasecounter}: #1}}       % display the title before the 
                                                % content of the environment is displayed 
}{
    \medskip
}



%**************************************************************
% Environment per ``sub sub use case''
%**************************************************************


\newcounter{subsubusecasecounter}             % define a counter
\setcounter{subsubusecasecounter}{0}          % set the counter to some initial value



\newenvironment{subsubusecase}[1]{
    \setcounter{subsubsubusecasecounter}{0}

    \refstepcounter{subsubusecasecounter}             % increment counter
    \protected@edef\@currentlabelname{\textbf{\usecasename \arabic{usecasecounter}.\arabic{subusecasecounter}.\arabic{subsubusecasecounter}}}% addition here

    % \def\@currentlabelname{\textbf{\usecasename \arabic{usecasecounter}}%
    \vspace{10pt}
    \par \noindent                              % start new paragraph
    {\large \textbf{\usecasename \arabic{usecasecounter}.\arabic{subusecasecounter}.\arabic{subsubusecasecounter}: #1}}       % display the title before the 
                                                % content of the environment is displayed 
    \medskip
}{
    \medskip
}


%**************************************************************
% Environment per ``sub sub sub use case''
%**************************************************************


\newcounter{subsubsubusecasecounter}             % define a counter
\setcounter{subsubsubusecasecounter}{0}          % set the counter to some initial value



\newenvironment{subsubsubusecase}[1]{
    \refstepcounter{subsubsubusecasecounter}             % increment counter
    \protected@edef\@currentlabelname{\textbf{\usecasename \arabic{usecasecounter}.\arabic{subusecasecounter}.\arabic{subsubusecasecounter}.\arabic{subsubsubusecasecounter}}}% addition here

    \vspace{10pt}
    \par \noindent                              % start new paragraph
    {\large \textbf{\usecasename \arabic{usecasecounter}.\arabic{subusecasecounter}.\arabic{subsubusecasecounter}.\arabic{subsubsubusecasecounter}: #1}}       % display the title before the 
                                                % content of the environment is displayed 
    \medskip
}{
    \medskip
}



%**************************************************************
% reset sub use case and sub sub use case counter
%**************************************************************
\newcommand{\resetcounter}{\setcounter{subusecasecounter}{0} \setcounter{subsubusecasecounter}{0}}

\setlength\epigraphwidth{8cm}
\setlength\epigraphrule{0pt}


\newcommand{\usecasename}{UC}

\newcommand{\usecaseactors}[1]{\textbf{\\Attori Principali:} #1. \vspace{4pt}}
\newcommand{\usecasepre}[1]{\textbf{\\Precondizioni:} #1. \vspace{4pt}}
\newcommand{\usecasedesc}[1]{\textbf{\\Descrizione:} #1. \vspace{4pt}}
\newcommand{\usecasepost}[1]{\textbf{\\Postcondizioni:} #1. \vspace{4pt}}
\newcommand{\usecasescenarioprincipale}[1]{\textbf{\\Scenario principale:} #1 \vspace{4pt}}
\newcommand{\usecasealt}[1]{\textbf{\\Scenario Alternativo:} #1. \vspace{4pt}}
\newcommand{\usecaseest}[1]{\textbf{\\Estensioni:} #1. \vspace{4pt}}

%**************************************************************
% Environment per ``namespace description''
%**************************************************************

\newenvironment{namespacedesc}{
    \vspace{10pt}
    \par \noindent                              % start new paragraph
    \begin{description} 
}{
    \end{description}
    \medskip
}

\newcommand{\classdesc}[2]{\item[\textbf{#1:}] #2}

\newcommand{\prospettoSettimanale}{
     % Personalizzare indicando in lista, i vari task settimana per settimana
     % sostituire a XX il totale ore della settimana
    \begin{itemize}
        \item \textbf{Prima Settimana (40 ore)}
        \begin{itemize}
            \item Incontro con persone coinvolte nel progetto per discutere i requisiti e le richieste relativamente al sistema da sviluppare;
            \item Verifica credenziali e strumenti di lavoro assegnati;
            \item  Ripasso Java Standard Edition e tool di sviluppo (IDE ecc.);
            \item Studio teorico dell’architettura a microservizi: passaggio da monolite a microservizi con pro e contro;
            \item Ripasso principi della buona programmazione (SOLID, CleanCode);
            \item Ripasso Java Standard Edition.
        \end{itemize}
        \item \textbf{Seconda Settimana - (40 ore)} 
        \begin{itemize}
            \item Studio teorico dell’architettura a microservizi: passaggio da monolite ad architetture a microservizi;
            \item Studio teorico dell’architettura a microservizi: Api Gateway, Service Discovery e Service Registry, Circuit Breaker e Saga Pattern;
            \item Studio Spring Core/Spring Boot.
        \end{itemize}
        \item \textbf{Terza Settimana - (40 ore)} 
        \begin{itemize}
            \item Studio servizi REST e framework Spring Data REST;
            \item Studio ORM, in particolare il framework Spring Data JPA.
        \end{itemize}
        \item \textbf{Quarta Settimana - (40 ore)} 
        \begin{itemize}
            \item Studio ORM, in particolare il framework Spring Data JPA.;
        \end{itemize}
        \item \textbf{Quinta Settimana - (40 ore)} 
        \begin{itemize}
            \item Studio della piattaforma SportWill esistente;
            \item Analisi nuova funzionalità da implementare.
        \end{itemize}
        \item \textbf{Sesta Settimana - (40 ore)} 
        \begin{itemize}
            \item Implementazione del nuovo servizio.
        \end{itemize}
        \item \textbf{Settima Settimana - (40 ore)} 
        \begin{itemize}
            \item Implementazione del nuovo servizio.
        \end{itemize}
        \item \textbf{Ottava Settimana - Conclusione (40 ore)} 
        \begin{itemize}
            \item Considerazioni e collaudi finali.
        \end{itemize}
    \end{itemize}
}


\newcommand{\obiettiviObbligatori}{
	 \item \underline{\textit{O01}}: Acquisizione competenze sulle tematiche sopra descritte;
	 \item \underline{\textit{O02}}: Capacità di raggiungere gli obiettivi richiesti in autonomia seguendo il cronoprogramma;
	 \item \underline{\textit{O03}}: Portare a termine l’implementazione dei microservizi richiesti con una percentuale di superamento pari al 80.
	 
}

\newcommand{\obiettiviDesiderabili}{
	 \item \underline{\textit{D01}}: Portare a termine l’implementazione dei microservizi richiesti con una percentuale di superamento pari al 100.
}

\newcommand{\obiettiviFacoltativi}{
	 \item \underline{\textit{F01}}: Utilizzo della containerizzazione per portare tutti i microservizi su Docker.
}