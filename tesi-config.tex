%**************************************************************
% file contenente le impostazioni della tesi
%**************************************************************

%**************************************************************
% Frontespizio
%**************************************************************

% Autore

\newcommand{\mappaturaemetodo}[2]{\textbf{Mappatura}: \texttt{/gruppi#1},
    \textbf{verbo HTTP}: #2}
\newcommand{\class}[1]{\textit{\textbf{#1}}}
\newcommand{\myparagraph}[1]{\paragraph{#1}\mbox{}\\}
\newcommand{\mysubparagraph}[1]{\subparagraph{#1}\mbox{}\vspace{3pt} \\}
\makeatletter
\newcounter{subsubparagraph}[subparagraph]
\renewcommand\thesubsubparagraph{%
    \thesubparagraph.\@arabic\c@subsubparagraph}
\newcommand\subsubparagraph{%
    \@startsection{subsubparagraph}    % counter
    {6} 			     % level
    {\parindent}		     % indent
    {3.25ex \@plus 1ex \@minus .2ex} % beforeskip
    {-1em}			     % afterskip
    {\normalfont\normalsize\bfseries}}
\newcommand\l@subsubparagraph{\@dottedtocline{6}{10em}{5em}}
\newcommand{\subsubparagraphmark}[1]{}
\makeatother

\definecolor{codegray}{gray}{0.9}
\newcommand{\code}[1]{\colorbox{codegray}{\texttt{#1}}}
\newcommand{\myName}{Lorenzo Matterazzo\xspace}
\newcommand{\myCompany}{Sync Lab S.r.l\xspace}

\newcommand{\myTitle}{Un applicativo web per la gestione di attività sportive:
    aggiunta di nuove funzionalità mediante Spring e Angular}
\newcommand{\productName}{\textit{\textbf{SportWill}}\xspace}
% Tipo di tesi                   
\newcommand{\myDegree}{Tesi di laurea triennale}

% Università             
\newcommand{\myUni}{Università degli Studi di Padova}

% Facoltà       
\newcommand{\myFaculty}{Corso di Laurea in Informatica}

% Dipartimento
\newcommand{\myDepartment}{Dipartimento di Matematica "Tullio Levi-Civita"}

% Titolo del relatore
\newcommand{\profTitle}{Prof.}

% Relatore
\newcommand{\myProf}{Paolo Baldan}

% Luogo
\newcommand{\myLocation}{Padova}

% Anno accademico
\newcommand{\myAA}{2020-2021}

% Data discussione
\newcommand{\myTime}{Dicembre 2021}

%**************************************************************
% Impostazioni di impaginazione
% see: http://wwwcdf.pd.infn.it/AppuntiLinux/a2547.htm
%**************************************************************

\setlength{\parindent}{14pt}   % larghezza rientro della prima riga
\setlength{\parskip}{0pt}   % distanza tra i paragrafi

%**************************************************************
% Impostazioni di biblatex
%**************************************************************
\bibliography{bibliografia} % database di biblatex 

\defbibheading{bibliography} {
    \cleardoublepage
    \phantomsection
    \addcontentsline{toc}{chapter}{\bibname}
    \chapter*{\bibname\markboth{\bibname}{\bibname}}
}

\setlength\bibitemsep{1.5\itemsep} % spazio tra entry

\DeclareBibliographyCategory{opere}
\DeclareBibliographyCategory{web}

\addtocategory{opere}{womak:lean-thinking}
\addtocategory{web}{site:agile-manifesto}

\defbibheading{opere}{\section*{Riferimenti bibliografici}}
\defbibheading{web}{\section*{Siti Web consultati}}

%**************************************************************
% Impostazioni di caption
%**************************************************************
\captionsetup{
    tableposition=top,
    figureposition=bottom,
    font=small,
    format=hang,
    labelfont=bf
}

%**************************************************************
% Impostazioni di glossaries
%**************************************************************

%**************************************************************
% Acronimi
%**************************************************************
\renewcommand{\acronymname}{Acronimi e abbreviazioni}

\newacronym[description={\glslink{apig}{Application Program Interface}}]
    {API}{API}{Application Program Interface}

\newacronym[description={\glslink{umlg}{Unified Modeling Language}}]
    {uml}{UML}{Unified Modeling Language}

\newacronym[description={\glslink{spag}{Single-Page Application}}]{spa}{SPA}{Single-Page Application}


\newacronym[description={\glslink{IPg}{Internet Protocol address}}]{IP}{IP}{Internet Protocol address}


\newacronym[first = {URL}]{URL}{URL}{
        Uniform Resource Locator}


\newacronym[description={\glslink{CRUDg}{Create Read Update Delete}}]
    {CRUD}{CRUD}{Create Read Update Delete}

    \newacronym[description={\glslink{JSPg}{JavaServer Pages}}]
    {JSP}{JSP}{JavaServer Pages}

\newacronym[description={\glslink{ICTg}{Information and Comunications Tecnology}}]
    {ICT}{ICT}{Information and Comunications Tecnology}
%**************************************************************
% Glossario
%**************************************************************
%\renewcommand{\glossaryname}{Glossario}

\newglossaryentry{apig}
{
    name=\glslink{api}{API},
    text=Application Program Interface,
    sort=api,
    description={in informatica con il termine \emph{Application Programming Interface API} (ing. interfaccia di programmazione di un'applicazione) si indica ogni insieme di procedure disponibili al programmatore, di solito raggruppate a formare un set di strumenti specifici per l'espletamento di un determinato compito all'interno di un certo programma. La finalità è ottenere un'astrazione, di solito tra l'hardware e il programmatore o tra software a basso e quello ad alto livello semplificando così il lavoro di programmazione}
}

\newglossaryentry{superset}
{
    name=\textit{superset},
    description={Un linguaggio di programmazione che contiene tutte le funzionalità di un determinato linguaggi, però ampliandolo o migliorandolo per includere anche altre funzionalità}
}
\newglossaryentry{endpoint}
{
    name=\textit{endpoint},
    description={Canale da cui le \gls{API} possono accedere alle risorse di cui hanno bisogno per eseguire la loro funzione}
}


\newglossaryentry{JSPg}
{
    name=JSP,
    description={documento di testo, scritto con una specifica sintassi, che rappresenta una pagina \textit{web} di contenuto parzialmente o totalmente dinamico. Elaborando la pagina JSP, il motore JSP produce dinamicamente la pagina HTML finale che verrà rappresentata nel \textit{browser} dell'utente}
}

\newglossaryentry{IPg}
{
    name=IP,
    description={codice numerico usato da tutti i dispositivi (\textit{computer}, \textit{server web}, stampanti, \textit{modem}) per navigare in Internet e per comunicare in una rete locale. Un indirizzo IP costituisce quindi la base per una trasmissione corretta delle informazioni dal mittente al ricevente.}
}

\newglossaryentry{spag}
{
    name=SPA,
    description={Una \textit{single-page application} (SPA) è un'applicazione \textit{web} o un sito \textit{web} che interagisce con l'utente riscrivendo dinamicamente la pagina \textit{web} corrente con nuovi dati dal server, invece del metodo predefinito di un \textit{browser} \textit{web} che carica intere nuove pagine.}
}

\newglossaryentry{umlg}
{
    name=\glslink{uml}{UML},
    text=UML,
    sort=uml,
    description={in ingegneria del software \emph{UML, Unified Modeling Language} (ing. linguaggio di modellazione unificato) è un linguaggio di modellazione e specifica basato sul paradigma object-oriented. L'\emph{UML} svolge un'importantissima funzione di ``lingua franca'' nella comunità della progettazione e programmazione a oggetti. Gran parte della letteratura di settore usa tale linguaggio per descrivere soluzioni analitiche e progettuali in modo sintetico e comprensibile a un vasto pubblico}
}
\newglossaryentry{microservizio}
{
    name={microservizio},
    description={},
    plural = {microservizi}
}

\newglossaryentry{transcompilazione}
{
    name={transcompilazione},
    description={Tipo di traduzione 
    che prende come input il codice sorgente di un programma scritto in un linguaggio di programmazione e produce un codice sorgente equivalente nello stesso o in un linguaggio di programmazione diverso.},
}


\newglossaryentry{framework}
{
    name={framework},
    description={}
}
\newglossaryentry{Spring}
{
    name={Spring},
    description={}
}

\newglossaryentry{CRUDg}
{
    name={CRUD},
    description={Operazioni di base che possono essere svolte su un database. In particolare,
    sono creazione (Create), lettura (Read), modifica/aggiornamento (Update) ed
    eliminazione (Delete).}
}

\newglossaryentry{API Gateway}
{
    name={API Gateway},
    description={}
}


\newglossaryentry{SRP}
{
    name=\textit{Single Responsibility Principle},
    description={Nella programmazione orientata agli oggetti, il principio di singola responsabilità afferma che ogni elemento di un programma deve avere una sola responsabilità, e che tale responsabilità debba essere interamente incapsulata dall'elemento stesso.}
}

\newglossaryentry{Eureka Server}
{
    name={Eureka Server},
    description={}
}


\newglossaryentry{will}
{
    name={\textit{will}},
    description={Intenzione di un utente di fare sport.}
}


\newglossaryentry{containerizzazione}
{
    name={containerizzazione},
    description={}
}

\newglossaryentry{Angular}
{
    name={Angular},
    description={}
}

\newglossaryentry{Trello}
{
    name={Trello},
    description={}
}

\newglossaryentry{ICTg}
{
    name={ICT},
    description={in informatica con il termine \textit{Information and Comunications Tecnology} 
    si indica l'insieme dei metodi e delle tecniche utilizzate nella trasmissione, ricezione 
    ed elaborazione di dati e informazioni.}
}


% Customize the format of the first use.  See the manual for details if
% you want to include more information here such as the definition.
\defglsdisplayfirst[\glsdefaulttype]{#1\glsfirstoccur}
 % database di termini
\makeglossaries

%**************************************************************
% Impostazioni di graphicx
%**************************************************************
\graphicspath{{immagini/}} % cartella dove sono riposte le immagini

%**************************************************************
% Impostazioni di hyperref
%**************************************************************
\hypersetup{
    %hyperfootnotes=false,
    %pdfpagelabels,
    %draft,	% = elimina tutti i link (utile per stampe in bianco e nero)
    colorlinks=true,
    linktocpage=true,
    pdfstartpage=1,
    pdfstartview=,
    % decommenta la riga seguente per avere link in nero (per esempio per la stampa in bianco e nero)
    %colorlinks=false, linktocpage=false, pdfborder={0 0 0}, pdfstartpage=1, pdfstartview=FitV,
    breaklinks=true,
    pdfpagemode=UseNone,
    pageanchor=true,
    pdfpagemode=UseOutlines,
    plainpages=false,
    bookmarksnumbered,
    bookmarksopen=true,
    bookmarksopenlevel=1,
    hypertexnames=true,
    pdfhighlight=/O,
    %nesting=true,
    %frenchlinks,
    urlcolor=webbrown,
    linkcolor=RoyalBlue,
    citecolor=webgreen,
    %pagecolor=RoyalBlue,
    %urlcolor=Black, linkcolor=Black, citecolor=Black, %pagecolor=Black,
    pdftitle={\myTitle},
    pdfauthor={\textcopyright\ \myName, \myUni, \myFaculty},
    pdfsubject={},
    pdfkeywords={},
    pdfcreator={pdfLaTeX},
    pdfproducer={LaTeX}
}

%**************************************************************
% Impostazioni di itemize
%**************************************************************
% \renewcommand{\labelitemi}{$\ast$}

%\renewcommand{\labelitemi}{$\bullet$}
%\renewcommand{\labelitemii}{$\cdot$}
%\renewcommand{\labelitemiii}{$\diamond$}
%\renewcommand{\labelitemiv}{$\ast$}

%**************************************************************
% Impostazioni di listings
%**************************************************************
\lstset{
    language=[LaTeX]Tex,%C++,
    keywordstyle=\color{RoyalBlue}, %\bfseries,
    basicstyle=\small\ttfamily,
    %identifierstyle=\color{NavyBlue},
    commentstyle=\color{Green}\ttfamily,
    stringstyle=\rmfamily,
    numbers=none, %left,%
    numberstyle=\scriptsize, %\tiny
    stepnumber=5,
    numbersep=8pt,
    showstringspaces=false,
    breaklines=true,
    frameround=ftff,
    frame=single
}

\lstdefinelanguage{json}{
basicstyle=\normalfont\ttfamily,
numbers=left,
captionpos=b,
numberstyle=\scriptsize,
stepnumber=1,
numbersep=8pt,
showstringspaces=false,
breaklines=true,
frame=lines,
backgroundcolor=\color{background},
literate=
*{0}{{{\color{numb}0}}}{1}
{1}{{{\color{numb}1}}}{1}
{2}{{{\color{numb}2}}}{1}
{3}{{{\color{numb}3}}}{1}
{4}{{{\color{numb}4}}}{1}
{5}{{{\color{numb}5}}}{1}
{6}{{{\color{numb}6}}}{1}
{7}{{{\color{numb}7}}}{1}
{8}{{{\color{numb}8}}}{1}
{9}{{{\color{numb}9}}}{1}
{:}{{{\color{punct}{:}}}}{1}
{,}{{{\color{punct}{,}}}}{1}
{\{}{{{\color{delim}{\{}}}}{1}
{\}}{{{\color{delim}{\}}}}}{1}
{[}{{{\color{delim}{[}}}}{1}
{]}{{{\color{delim}{]}}}}{1},
}

\lstdefinelanguage{TypeScript}{
    keywords={typeof, new, true, false, catch, function, return, null, catch,
            switch, var, if, in, while, do, else, case, break},
    keywordstyle=\color{blue}\bfseries,
    ndkeywords={class, export, boolean, throw, implements, import, this, private,
        },
    ndkeywordstyle=\color{darkgray}\bfseries,
    identifierstyle=\color{black},
    sensitive=false,
    comment=[l]{//},
    morecomment=[s]{/*}{*/},
    commentstyle=\color{purple}\ttfamily,
    stringstyle=\color{red}\ttfamily,
    morestring=[b]',
    morestring=[b]",
    extendedchars=true,
    basicstyle=\footnotesize\ttfamily,
    showstringspaces=false,
    showspaces=false,
    numbers=left,
    numberstyle=\footnotesize,
    numbersep=9pt,
    tabsize=2,
    breaklines=true,
    showtabs=false,
    captionpos=b
}

\lstdefinestyle{Java}{
language = Java ,
% frame = trBL ,
firstnumber = 1,
tabsize = 4, %% set tab space width
showstringspaces = false, %% prevent space marking in strings, string is defined as the text that is generally printed directly to the console
numbers = left, %% display line numbers on the left
commentstyle = \color{mygreen}, %% set comment color
keywordstyle = \color{blue}, %% set keyword color,
numberstyle = \color{mygreen},
morekeywords = {@Component, @RestController, @Slf4j, @RequestMapping,
        @CrossOrigin, @Override, @Autowired, EurekaClient, Mono, Void, Void, Config,
        String, ServerHttpResponse, typeof, new, true, false, catch, function, return,
        null, catch, switch, var, if, in, while, do, else, case, break, class, export,
        boolean, constructor, throw, implements,let,of, import, this, private},
captionpos=b,
stepnumber=1,			 % the step between two line-numbers. If it's 1, each line will be numbered
stringstyle = \color{red}, %% set string color
rulecolor = \color{black}, %% set frame color to avoid being affected by text color
morestring=[s]{`}{'},
basicstyle = \small \ttfamily , %% set listing font and size
breaklines = true, %% enable line breaking
numberstyle = \tiny,
literate=%
    *{0}{{{\color{mygreen}{0}}}}1
{1}{{{\color{mygreen}{1}}}}1
{2}{{{\color{mygreen}{2}}}}1
{3}{{{\color{mygreen}{3}}}}1
{4}{{{\color{mygreen}{4}}}}1
{5}{{{\color{mygreen}{5}}}}1
{6}{{{\color{mygreen}{6}}}}1
{7}{{{\color{mygreen}{7}}}}1
{8}{{{\color{mygreen}{8}}}}1
{9}{{{\color{mygreen}{9}}}}1
}

\lstdefinestyle{Docker}{
    frame=tb,
    language=Go,
    backgroundcolor=\color{white},   % choose the background color; you must add \usepackage{color} or \usepackage{xcolor}; should come as last argument
    breakatwhitespace=false,	   % sets if automatic breaks should only happen at whitespace
    breaklines=true,		   % sets automatic line breaking
    captionpos=b, 		   % sets the caption-position to bottom
    commentstyle=\color{mygreen},    % comment style
    deletekeywords={//, /},	     % if you want to delete keywords from the given language
    escapechar=^,
    %   escapeinside={\%*}{*)},          % if you want to add LaTeX within your code
    extendedchars=true,		   % lets you use non-ASCII characters; for 8-bits encodings only, does not work with UTF-8
    %   firstnumber=1000,                % start line enumeration with line 1000
    frame=single, 		   % adds a frame around the code
    keepspaces=true,		   % keeps spaces in text, useful for keeping indentation of code (possibly needs columns=flexible)
    keywordstyle=\color{bluecode},       % keyword style
    morekeywords={*,FROM, ARG, COPY, ENTRYPOINT},
    % if you want to add more keywords to the set
    numbers=left, 		   % where to put the line-numbers; possible values are (none, left, right)
    numbersep=5pt,		   % how far the line-numbers are from the code
    numberstyle=\tiny\color{mygray}, % the style that is used for the line-numbers
    rulecolor=\color{black},	   % if not set, the frame-color may be changed on line-breaks within not-black text (e.g. comments (green here))
    showspaces=false,		   % show spaces everywhere adding particular underscores; it overrides 'showstringspaces'
    showstringspaces=false,	   % underline spaces within strings only
    showtabs=false,		   % show tabs within strings adding particular underscores
    stepnumber=1, 		   % the step between two line-numbers. If it's 1, each line will be numbered
    stringstyle=\color{mymauve},	   % string literal style
    tabsize=2,			   % sets default tabsize to 2 spaces
    title=\lstname		   % show the filename of files included with \lstinputlisting; also try caption instead of title
}
\addto\captionsitalian{%
    \renewcommand{\lstlistingname}{Codice}%
    \renewcommand\lstlistlistingname{Elenco dei codici}}

\makeatletter

% here is a macro expanding to the name of the language
% (handy if you decide to change it further down the road)
\newcommand\language@yaml{yaml}

\expandafter\expandafter\expandafter\lstdefinelanguage
\expandafter{\language@yaml}
{
keywords={true,false,null,y,n},
captionpos=b,
keywordstyle=\color{darkgray}\bfseries,
basicstyle=\YAMLkeystyle,
numbers=left,		     % where to put the line-numbers; possible values are (none, left, right)
numbersep=5pt,		     % how far the line-numbers are from the code
numberstyle=\tiny\color{mygray},
% assuming a key comes first
sensitive=false,
comment=[l]{\#},
% morecomment=[s]{/*}{*/},
commentstyle=\color{purple}\ttfamily,
stringstyle=\YAMLvaluestyle\ttfamily,
escapechar=^,
moredelim=[l][\color{orange}]{\&},
moredelim=[l][\color{magenta}]{*},
moredelim=**[il][\YAMLcolonstyle{:}\YAMLvaluestyle]{:},
% switch to value style at :
morestring=[b]',
morestring=[b]",
morestring=[s]{`}{'},
stepnumber=1,
literate =	  {---}{{\ProcessThreeDashes}}3
{>}{{\textcolor{red}\textgreater}}1
{|}{{\textcolor{red}\textbar}}1
{\ -\ }{{\mdseries\ -\ }}3,
}

% switch to key style at EOL
\lst@AddToHook{EveryLine}{\ifx\lst@language\language@yaml\YAMLkeystyle\fi}
\makeatother

\newcommand\ProcessThreeDashes{\llap{\color{cyan}\mdseries-{-}-}}

%**************************************************************
% Impostazioni di xcolor
%**************************************************************
\definecolor{webgreen}{rgb}{0,.5,0}
\definecolor{webbrown}{rgb}{.6,0,0}

\definecolor{maincolor}{RGB}{155, 0, 20}
\definecolor{antimaincolor}{RGB}{255, 255, 255}
\definecolor{color1}{RGB}{210, 210, 210}
\definecolor{color2}{RGB}{230, 230, 230}
\definecolor{bluecode}{RGB}{86, 156, 214}

\definecolor{mauve}{rgb}{0.58,0,0.82}
\definecolor{mymauve}{rgb}{0.58,0,0.82}

\definecolor{mygreen}{rgb}{0,0.6,0}
\definecolor{mygray}{rgb}{0.5,0.5,0.5}

\colorlet{punct}{red!60!black}
\definecolor{background}{HTML}{EEEEEE}
\definecolor{delim}{RGB}{20,105,176}
\colorlet{numb}{magenta!60!black}

\definecolor{lightgray}{rgb}{.9,.9,.9}
\definecolor{darkgray}{rgb}{.4,.4,.4}
\definecolor{purple}{rgb}{0.65, 0.12, 0.82}

\newcommand\YAMLcolonstyle{\color{red}\mdseries}
\newcommand\YAMLkeystyle{\color{black}}
\newcommand\YAMLvaluestyle{\color{blue}\mdseries}

%**************************************************************
% Altro
%**************************************************************

\newcommand{\omissis}{[\dots\negthinspace]} % produce [...]

% eccezioni all'algoritmo di sillabazione
\hyphenation
{
    ma-cro-istru-zio-ne
    gi-ral-din
}

\newcommand{\sectionname}{sezione}
\addto\captionsitalian{\renewcommand{\figurename}{Figura}
    \renewcommand{\tablename}{Tabella}}

\newcommand{\glsfirstoccur}{\ap{{[g]}}}

\newcommand{\intro}[1]{\emph{\textsf{#1}}}

%**************************************************************
% Environment per ``rischi''
%**************************************************************
\newcounter{riskcounter}		% define a counter
\setcounter{riskcounter}{0}		% set the counter to some initial value

%%%% Parameters
% #1: Title
\newenvironment{risk}[1]{
    \refstepcounter{riskcounter}	% increment counter
    \par \noindent			% start new paragraph
    \textbf{\arabic{riskcounter}. #1}	% display the title before the 
    % content of the environment is displayed 
}{
    \par\medskip
}

\newcommand{\riskname}{Rischio}

\newcommand{\riskdescription}[1]{\textbf{\\Descrizione:} #1.}

\newcommand{\risksolution}[1]{\textbf{\\Piano di contingenza:} #1.}

%**************************************************************
% Environment per ``use case''
%**************************************************************
\newcounter{usecasecounter}		% define a counter
\setcounter{usecasecounter}{0}		% set the counter to some initial value

\makeatletter

%%%% Parameters
% #1: Nome
\newenvironment{usecase}[1]{
    \resetcounter
    \renewcommand{\theusecasecounter}{\usecasename #1}
    % this is where the display of                                                         % the counter is overwritten/modified
    \refstepcounter{usecasecounter}
    % increment counter
    \protected@edef\@currentlabelname{\textbf{\usecasename
            \arabic{usecasecounter}}}% addition here
    \vspace{10pt}
    \par \noindent
    % start new paragraph
    {\large \textbf{\usecasename \arabic{usecasecounter}: #1}}
    % display the title before the 

    % content of the environment is displayed 
    \medskip
}{
    \medskip
}

%**************************************************************
% titlespacing settings
%**************************************************************
\titlespacing*{\section}{0pt}{1.1\baselineskip}{\baselineskip}
\titlespacing*{\subsection}{0pt}{1.1\baselineskip}{\baselineskip}
% \titlespacing*{\subparagraph}{0pt}{1.1\baselineskip}{\baselineskip}

%**************************************************************
% titleformat settings
%**************************************************************
\titleformat{\subparagraph}[runin]{\normalfont\normalsize\bfseries}{}{0pt}{}
\titleformat{\paragraph}[runin]{\normalfont\normalsize\bfseries}{}{0pt}{}

%**************************************************************
% Environment per ``sub use case''
%**************************************************************
\newcounter{subusecasecounter}		   % define a counter
\setcounter{subusecasecounter}{0}	   % set the counter to some initial value

\newenvironment{subusecase}[1]{
    \setcounter{subsubusecasecounter}{0}
    \refstepcounter{subusecasecounter}
    % increment counter
    \protected@edef\@currentlabelname{\textbf{\usecasename
            \arabic{usecasecounter}.\arabic{subusecasecounter}}}% addition here

    % \def\@currentlabelname{\textbf{\usecasename \arabic{usecasecounter}}%
    \vspace{10pt}
    \noindent				   % start new paragraph
    {\large \textbf{\usecasename
            \arabic{usecasecounter}.\arabic{subusecasecounter}: #1}}
    % display the title before the 

    % content of the environment is displayed 
}{
    \medskip
}

%**************************************************************
% Environment per ``sub sub use case''
%**************************************************************

\newcounter{subsubusecasecounter}
% define a counter
\setcounter{subsubusecasecounter}{0}
% set the counter to some initial value

\newenvironment{subsubusecase}[1]{
    \setcounter{subsubsubusecasecounter}{0}

    \refstepcounter{subsubusecasecounter}
    % increment counter
    \protected@edef\@currentlabelname{\textbf{\usecasename
            \arabic{usecasecounter}.\arabic{subusecasecounter}.\arabic{subsubusecasecounter}}}% addition here

    % \def\@currentlabelname{\textbf{\usecasename \arabic{usecasecounter}}%
    \vspace{10pt}
    \par \noindent
    % start new paragraph
    {\large \textbf{\usecasename
            \arabic{usecasecounter}.\arabic{subusecasecounter}.\arabic{subsubusecasecounter}:
            #1}}	   % display the title before the 

    % content of the environment is displayed 
    \medskip
}{
    \medskip
}

%**************************************************************
% Environment per ``sub sub sub use case''
%**************************************************************

\newcounter{subsubsubusecasecounter}
% define a counter
\setcounter{subsubsubusecasecounter}{0}
% set the counter to some initial value

\newenvironment{subsubsubusecase}[1]{
    \refstepcounter{subsubsubusecasecounter}
    % increment counter
    \protected@edef\@currentlabelname{\textbf{\usecasename
            \arabic{usecasecounter}.\arabic{subusecasecounter}.\arabic{subsubusecasecounter}.\arabic{subsubsubusecasecounter}}}% addition here

    \vspace{10pt}
    \par \noindent
    % start new paragraph
    {\large \textbf{\usecasename
            \arabic{usecasecounter}.\arabic{subusecasecounter}.\arabic{subsubusecasecounter}.\arabic{subsubsubusecasecounter}:
            #1}}	   % display the title before the 

    % content of the environment is displayed 
    \medskip
}{
    \medskip
}

%**************************************************************
% reset sub use case and sub sub use case counter
%**************************************************************
\newcommand{\resetcounter}{\setcounter{subusecasecounter}{0}
    \setcounter{subsubusecasecounter}{0}}

\setlength\epigraphwidth{8cm}
\setlength\epigraphrule{0pt}

\newcommand{\usecasename}{UC}

\newcommand{\usecaseactors}[1]{\textbf{\\Attori Principali:} #1. \vspace{4pt}}
\newcommand{\usecasepre}[1]{\textbf{\\Precondizioni:} #1. \vspace{4pt}}
\newcommand{\usecasedesc}[1]{\textbf{\\Descrizione:} #1. \vspace{4pt}}
\newcommand{\usecasepost}[1]{\textbf{\\Postcondizioni:} #1. \vspace{4pt}}
\newcommand{\usecasescenarioprincipale}[1]{\textbf{\\Scenario principale:} #1
    \vspace{4pt}}
\newcommand{\usecasealt}[1]{\textbf{\\Scenario Alternativo:} #1. \vspace{4pt}}
\newcommand{\usecaseest}[1]{\textbf{\\Estensioni:} #1. \vspace{4pt}}

%**************************************************************
% Environment per ``namespace description''
%**************************************************************

\newenvironment{namespacedesc}{
    \vspace{10pt}
    \par \noindent
    % start new paragraph
    \begin{description}
        }{
    \end{description}
    \medskip
}

\newcommand{\classdesc}[2]{\item[\textbf{#1:}] #2}

\newcommand{\prospettoSettimanale}{
    % Personalizzare indicando in lista, i vari task settimana per settimana
    % sostituire a XX il totale ore della settimana
    \begin{itemize}
        \item \textbf{Prima Settimana (40 ore)}
              \begin{itemize}
                  \item Incontro con persone coinvolte nel progetto per discutere i
                        requisiti e le richieste relativamente al sistema da sviluppare;
                  \item Verifica credenziali e strumenti di lavoro assegnati;
                  \item  Ripasso Java Standard Edition e tool di sviluppo (IDE ecc.);
                  \item Studio teorico dell’architettura a \glspl{microservizio}:
                        passaggio da monolite a \glspl{microservizio} con pro e contro;
                  \item Ripasso principi della buona programmazione (SOLID,
                        CleanCode);
                  \item Ripasso Java Standard Edition.
              \end{itemize}
        \item \textbf{Seconda Settimana - (40 ore)}
              \begin{itemize}
                  \item Studio teorico dell’architettura a \glspl{microservizio}:
                        passaggio da monolite ad architetture a \glspl{microservizio};
                  \item Studio teorico dell’architettura a \glspl{microservizio}: Api
                        Gateway, Service Discovery e Service Registry, Circuit Breaker e Saga Pattern;
                  \item Studio Spring Core/Spring Boot.
              \end{itemize}
        \item \textbf{Terza Settimana - (40 ore)}
              \begin{itemize}
                  \item Studio servizi REST e \gls{framework} Spring Data REST;
                  \item Studio ORM, in particolare il \gls{framework} Spring Data
                        JPA.
              \end{itemize}
        \item \textbf{Quarta Settimana - (40 ore)}
              \begin{itemize}
                  \item Studio ORM, in particolare il \gls{framework} Spring Data
                        JPA.;
              \end{itemize}
        \item \textbf{Quinta Settimana - (40 ore)}
              \begin{itemize}
                  \item Studio della piattaforma \productName esistente;
                  \item Analisi nuova funzionalità da implementare.
              \end{itemize}
        \item \textbf{Sesta Settimana - (40 ore)}
              \begin{itemize}
                  \item Implementazione del nuovo servizio.
              \end{itemize}
        \item \textbf{Settima Settimana - (40 ore)}
              \begin{itemize}
                  \item Implementazione del nuovo servizio.
              \end{itemize}
        \item \textbf{Ottava Settimana - Conclusione (40 ore)}
              \begin{itemize}
                  \item Considerazioni e collaudi finali.
              \end{itemize}
    \end{itemize}
}

\newcommand{\obiettiviObbligatori}{
    \item \underline{\textit{O01}}: Acquisizione competenze sulle
    tematiche sopra descritte;
    \item \underline{\textit{O02}}: Capacità di raggiungere gli obiettivi
    richiesti in autonomia seguendo il cronoprogramma;
    \item \underline{\textit{O03}}: Portare a termine l’implementazione
    dei \glspl{microservizio} richiesti con una percentuale di superamento pari al
    80.

}

\newcommand{\obiettiviDesiderabili}{
    \item \underline{\textit{D01}}: Portare a termine l’implementazione
    dei \glspl{microservizio} richiesti con una percentuale di superamento pari al
    100.
}

\newcommand{\obiettiviFacoltativi}{
    \item \underline{\textit{F01}}: Utilizzo della
    \gls{containerizzazione} per portare tutti i \glspl{microservizio} su Docker.
}