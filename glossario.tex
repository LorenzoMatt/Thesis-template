
%**************************************************************
% Acronimi
%**************************************************************
\renewcommand{\acronymname}{Acronimi e abbreviazioni}

\newacronym[description={\glslink{apig}{Application Program Interface}}]
    {API}{API}{Application Program Interface}

\newacronym[description={\glslink{umlg}{Unified Modeling Language}}]
    {uml}{UML}{Unified Modeling Language}

\newacronym[description={\glslink{spag}{Single-Page Application}}]{spa}{SPA}{Single-Page Application}


\newacronym[first={IP}, description={\glslink{IPg}{Internet Protocol address}}]{IP}{IP}{Internet Protocol address}


\newacronym[first = {URL}]{URL}{URL}{
        Uniform Resource Locator}


\newacronym[description={\glslink{CRUDg}{Create Read Update Delete}}]
    {CRUD}{CRUD}{Create Read Update Delete}

    \newacronym[description={\glslink{JSPg}{JavaServer Pages}}]
    {JSP}{JSP}{JavaServer Pages}

\newacronym[description={\glslink{ICTg}{Information and Comunications Tecnology}}]
    {ICT}{ICT}{Information and Comunications Tecnology}


\newacronym[description={\glslink{DAOg}{Data Access Object}}]
{DAO}{DAO}{Data Access Object}

\newacronym[first={JPA},description={\glslink{JPAg}{}}]
{JPA}{JPA}{Java Persistence API}


%**************************************************************
% Glossario
%**************************************************************
%\renewcommand{\glossaryname}{Glossario}

\newglossaryentry{apig}
{
    name=\glslink{api}{API},
    text=Application Program Interface,
    sort=api,
    description={in informatica con il termine \emph{Application Programming Interface API} (ing. interfaccia di programmazione di un'applicazione) si indica ogni insieme di procedure disponibili al programmatore, di solito raggruppate a formare un set di strumenti specifici per l'espletamento di un determinato compito all'interno di un certo programma. La finalità è ottenere un'astrazione, di solito tra l'hardware e il programmatore o tra software a basso e quello ad alto livello semplificando così il lavoro di programmazione}
}
\newglossaryentry{DAOg}
{
    name=DAO,
    description={\textit{pattern} che consente di isolare l'\textit{application}/\textit{business layer} dal \textit{persistence layer} (di solito un database relazionale, ma potrebbe essere qualsiasi altro meccanismo di persistenza)}
}



\newglossaryentry{superset}
{
    name=\textit{superset},
    description={Un linguaggio di programmazione che contiene tutte le funzionalità di un determinato linguaggi, però ampliandolo o migliorandolo per includere anche altre funzionalità}
}
\newglossaryentry{endpoint}
{
    name=\textit{endpoint},
    description={Canale da cui le \gls{API} possono accedere alle risorse di cui hanno bisogno per eseguire la loro funzione}
}

\newglossaryentry{JPAg}
{
    name=JPA,
    description={\textit{framework} che offre delle \gls{API} per aiutare gli sviluppatori nelle operazioni di persistenza dei dati su un \textit{database} relazionale}
}


\newglossaryentry{EntityManager}
{
    name=EntityManager,
    description={API che gestisce il ciclo di vita delle istanze di entità}
}

\newglossaryentry{container}
{
    name=container,
    description={formato di creazione dei pacchetti che racchiude tutto il codice e le dipendenze di un'applicazione in un formato \textit{standard} che ne consente l'esecuzione rapida e affidabile in tutti gli ambienti di elaborazione.\\ Ogni container Docker ha il proprio \textit{file system}, il proprio \textit{stack} di rete (quindi il proprio indirizzo \gls{IP}), il proprio spazio di elaborazione e limitazioni di risorse definite per CPU e memoria}
}

\newglossaryentry{JSPg}
{
    name=JSP,
    description={documento di testo, scritto con una specifica sintassi, che rappresenta una pagina \textit{web} di contenuto parzialmente o totalmente dinamico. Elaborando la pagina JSP, il motore JSP produce dinamicamente la pagina HTML finale che verrà rappresentata nel \textit{browser} dell'utente}
}

\newglossaryentry{IPg}
{
    name=IP,
    description={codice numerico usato da tutti i dispositivi (\textit{computer}, \textit{server web}, stampanti, \textit{modem}) per navigare in Internet e per comunicare in una rete locale. Un indirizzo IP costituisce quindi la base per una trasmissione corretta delle informazioni dal mittente al ricevente.}
}

\newglossaryentry{spag}
{
    name=SPA,
    description={Una \textit{single-page application} (SPA) è un'applicazione \textit{web} o un sito \textit{web} che interagisce con l'utente riscrivendo dinamicamente la pagina \textit{web} corrente con nuovi dati dal server, invece del metodo predefinito di un \textit{browser} \textit{web} che carica intere nuove pagine.}
}

\newglossaryentry{umlg}
{
    name=\glslink{uml}{UML},
    text=UML,
    sort=uml,
    description={in ingegneria del software \emph{UML, Unified Modeling Language} (ing. linguaggio di modellazione unificato) è un linguaggio di modellazione e specifica basato sul paradigma object-oriented. L'\emph{UML} svolge un'importantissima funzione di ``lingua franca'' nella comunità della progettazione e programmazione a oggetti. Gran parte della letteratura di settore usa tale linguaggio per descrivere soluzioni analitiche e progettuali in modo sintetico e comprensibile a un vasto pubblico}
}
\newglossaryentry{microservizio}
{
    name={microservizio},
    description={},
    plural = {microservizi}
}

\newglossaryentry{transcompilazione}
{
    name={transcompilazione},
    description={Tipo di traduzione 
    che prende come input il codice sorgente di un programma scritto in un linguaggio di programmazione e produce un codice sorgente equivalente nello stesso o in un linguaggio di programmazione diverso.},
}


\newglossaryentry{framework}
{
    name={framework},
    description={}
}
\newglossaryentry{Spring}
{
    name={Spring},
    description={}
}

\newglossaryentry{CRUDg}
{
    name={CRUD},
    description={Operazioni di base che possono essere svolte su un database. In particolare,
    sono creazione (Create), lettura (Read), modifica/aggiornamento (Update) ed
    eliminazione (Delete).}
}


\newglossaryentry{SRP}
{
    name=\textit{Single Responsibility Principle},
    description={Nella programmazione orientata agli oggetti, il principio di singola responsabilità afferma che ogni elemento di un programma deve avere una sola responsabilità, e che tale responsabilità debba essere interamente incapsulata dall'elemento stesso.}
}

\newglossaryentry{Eureka Server}
{
    name={Eureka Server},
    description={}
}


\newglossaryentry{will}
{
    name={\textit{will}},
    description={Intenzione di un utente di fare sport.}
}


\newglossaryentry{containerizzazione}
{
    name={containerizzazione},
    description={}
}

\newglossaryentry{Angular}
{
    name={Angular},
    description={}
}

\newglossaryentry{Trello}
{
    name={Trello},
    description={}
}

\newglossaryentry{ICTg}
{
    name={ICT},
    description={in informatica con il termine \textit{Information and Comunications Tecnology} 
    si indica l'insieme dei metodi e delle tecniche utilizzate nella trasmissione, ricezione 
    ed elaborazione di dati e informazioni.}
}


% Customize the format of the first use.  See the manual for details if
% you want to include more information here such as the definition.
\defglsdisplayfirst[\glsdefaulttype]{#1\glsfirstoccur}
