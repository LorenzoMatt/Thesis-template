%**************************************************************
% Acronimi
%**************************************************************
\renewcommand{\acronymname}{Acronimi e abbreviazioni}

\newacronym[description={\glslink{apig}{Application Program Interface}}]
{API}{API}{Application Program Interface}

\newacronym[description={\glslink{umlg}{Unified Modeling Language}}]
{uml}{UML}{Unified Modeling Language}

\newacronym[description={\glslink{spag}{Single-Page
                Application}}]{spa}{SPA}{Single-Page Application}

\newacronym[first={IP}, description={\glslink{IPg}{Internet Protocol
                address}}]{IP}{IP}{Internet Protocol address}

\newacronym[first = {URL}]{URL}{URL}{
    Uniform Resource Locator}

\newacronym[description={\glslink{CRUDg}{Create Read Update Delete}}]
{CRUD}{CRUD}{Create Read Update Delete}

\newacronym[description={\glslink{JSPg}{JavaServer Pages}}]
{JSP}{JSP}{JavaServer Pages}

\newacronym[description={\glslink{ICTg}{Information and Comunications
                Tecnology}}]
{ICT}{ICT}{Information and Comunications Tecnology}

\newacronym[description={\glslink{DAOg}{Data Access Object}}]
{DAO}{DAO}{Data Access Object}

\newacronym[first={JPA},description={\glslink{JPAg}{Java Persistence API}}]
{JPA}{JPA}{Java Persistence API}

%**************************************************************
% Glossario
%**************************************************************
%\renewcommand{\glossaryname}{Glossario}

\newglossaryentry{apig}
{
    name=\glslink{API}{API},
    text=Application Program Interface,
    sort=api,
    description={Insieme di procedure disponibili al programmatore, di solito
            raggruppate a formare un \textit{set} di strumenti specifici per
            l'espletamento di un
            determinato compito all'interno di un certo programma. La finalità
            è ottenere
            un'astrazione, di solito tra l'\textit{hardware} e il programmatore
            o tra
            \textit{software} a
            basso e quello ad alto livello semplificando così il lavoro di
            programmazione}
}
\newglossaryentry{DAOg}
{
    name=DAO,
    description={\textit{Pattern} che consente di isolare il \textit{business
                layer} dal \textit{persistence layer} (di solito un
            \textit{database}
            relazionale, ma
            potrebbe essere qualsiasi altro meccanismo di persistenza)}
}

% description={\textit{pattern} che consente di isolare l'\textit{application}/\textit{business layer} dal \textit{persistence layer} (di solito un database relazionale, ma potrebbe essere qualsiasi altro meccanismo di persistenza)}

\newglossaryentry{superset}
{
    name=\textit{superset},
    description={Un linguaggio di programmazione che contiene tutte le
            funzionalità di un determinato linguaggi, però ampliandolo o
            migliorandolo per
            includere anche altre funzionalità}
}
\newglossaryentry{endpoint}
{
    name=\textit{endpoint},
    description={Canale da cui le \gls{API} possono accedere alle risorse di
            cui hanno bisogno per eseguire la loro funzione}
}

\newglossaryentry{JPAg}
{
    name=JPA,
    description={\Gls{framework} che offre delle \gls{API} per aiutare gli
            sviluppatori nelle operazioni di persistenza dei dati su un
            \textit{database}
            relazionale}
}

\newglossaryentry{EntityManager}
{
    name=EntityManager,
    description={API che gestisce il ciclo di vita delle istanze di entità in
            un \textit{database}}
}

\newglossaryentry{container}
{
    name=\textit{container},
    description={Formato di creazione dei pacchetti che racchiude tutto il
            codice e le dipendenze di un'applicazione in un formato
            \textit{standard} che
            ne consente l'esecuzione rapida e affidabile in tutti gli ambienti
            di
            elaborazione.\\ Ogni \textit{container} Docker ha il proprio
            \textit{file
                system}, il
            proprio \textit{stack} di rete (quindi il proprio indirizzo
            \gls{IP}), il
            proprio spazio di elaborazione e limitazioni di risorse definite
            per CPU e
            memoria}
}

\newglossaryentry{JSPg}
{
    name=JSP,
    description={Documento di testo, scritto con una specifica sintassi, che
            rappresenta una pagina \textit{web} di contenuto parzialmente o
            totalmente
            dinamico. Elaborando la pagina JSP, il motore JSP produce
            dinamicamente la
            pagina HTML finale che verrà rappresentata nel \textit{browser}
            dell'utente}
}

\newglossaryentry{IPg}
{
    name=IP,
    description={Codice numerico usato da tutti i dispositivi
            (\textit{computer}, \textit{server web}, stampanti, \textit{modem})
            per
            navigare in Internet e per comunicare in una rete locale. Un
            indirizzo IP
            costituisce quindi la base per una trasmissione corretta delle
            informazioni dal
            mittente al ricevente}
}

\newglossaryentry{spag}
{
    name=SPA,
    description={Una \textit{single-page application} (SPA) è un'applicazione
            \textit{web} o un sito \textit{web} che interagisce con l'utente
            riscrivendo
            dinamicamente la pagina \textit{web} corrente con nuovi dati dal
            \textit{server}, invece del metodo predefinito di un
            \textit{browser}
            \textit{web} che carica intere nuove pagine}
}

\newglossaryentry{umlg}
{
    name=\glslink{uml}{UML},
    text=UML,
    sort=uml,
    description={Linguaggio di modellazione basato sul paradigma
            \textit{object-oriented}. L'\emph{UML} svolge un'importantissima
            funzione di
            \enquote*{lingua franca} nella comunità della progettazione e
            programmazione a
            oggetti. Gran parte della letteratura di settore usa tale
            linguaggio per
            descrivere soluzioni analitiche e progettuali in modo sintetico e
            comprensibile
            a un vasto pubblico}
}
\newglossaryentry{microservizio}
{
    name={microservizio},
    description={Approccio per sviluppare e organizzare l'architettura dei
            \textit{software} secondo cui quest’ultimi sono composti di servizi
            indipendenti di piccole dimensioni che comunicano tra loro tramite
            \gls{API}
            ben definite},
    plural = {microservizi}
}

\newglossaryentry{transcompilazione}
{
    name={transcompilazione},
    description={Tipo di traduzione
            che prende come \textit{input} il codice sorgente di un programma
            scritto
            in un linguaggio di programmazione e produce un codice sorgente
            equivalente
            nello stesso o in un linguaggio di programmazione diverso},
}

\newglossaryentry{framework}
{
    name={\textit{framework}},
    description={Piattaforma che funge da strato intermedio tra un sistema
            operativo e il \textit{software} che lo utilizza}
}

\newglossaryentry{CRUDg}
{
    name={CRUD},
    description={Operazioni di base che possono essere svolte su un
            \textit{database}. In particolare,
            sono creazione (Create), lettura (Read), modifica/aggiornamento
            (Update) ed
            eliminazione (Delete)}
}

\newglossaryentry{SRP}
{
    name=\textit{Single Responsibility Principle},
    description={Nella programmazione orientata agli oggetti, il \textit{single
                responsability principle} afferma che ogni elemento di un
            programma deve avere
            una
            sola responsabilità, e che tale responsabilità debba essere
            interamente
            incapsulata dall'elemento stesso}
}

\newglossaryentry{Eureka Server}
{
    name={Eureka Server},
    description={Applicazione che contiene le informazioni su tutte le
            applicazioni \textit{client-service}. Ogni \gls{microservizio} si
            registra nel
            \textit{server} Eureka ed esso conosce tutte le
            applicazioni in esecuzione su ciascuna porta e indirizzo \gls{IP}}
}

\newglossaryentry{will}
{
    name={\textit{will}},
    description={Intenzione di un utente di fare sport}
}

\newglossaryentry{screen reader}
{
    name={\textit{screen reader}},
    description={\textit{Software} che identifica ed interpreta il testo
            mostrato sullo schermo di un \textit{computer}, presentandolo
            tramite sintesi
            vocale o attraverso un \textit{display braille}. Sono utilizzati
            prevalentemente da persone con problemi di vista}
}

\newglossaryentry{containerizzazione}
{
    name={\textit{containerizzazione}},
    description={Approccio allo sviluppo del \textit{software} in cui
            un'applicazione o un servizio, le relative dipendenze e la
            corrispondente
            configurazione (astratti come file manifesto della distribuzione)
            sono inclusi
            in uno stesso pacchetto sotto forma di immagine del
            \gls{container}.
            L'applicazione inclusa nel \gls{container} può essere testata come
            unità e
            distribuita come istanza dell'immagine del \gls{container} al
            sistema operativo
            \textit{host}}
}

\newglossaryentry{Trello}
{
    name={Trello},
    description={\textit{Web app} che serve per gestire il/i team, gestire
            \textit{task}, programmare \textit{macro} e \textit{micro}
            attività, gestire
            l'esecuzione dei progetti, controllarne l'andamento e far sì che le
            varie
            azioni sui progetti o sul cliente, si intreccino in modo fluido}
}

\newglossaryentry{ICTg}
{
    name={ICT},
    description={Insieme dei metodi e delle tecniche utilizzate nella
            trasmissione, ricezione ed elaborazione di dati e informazioni}
}

% Customize the format of the first use.  See the manual for details if
% you want to include more information here such as the definition.
\defglsdisplayfirst[\glsdefaulttype]{#1\glsfirstoccur}