% !TEX encoding = UTF-8
% !TEX TS-program = pdflatex
% !TEX root = ../tesi.tex

%**************************************************************
\chapter{Progettazione e codifica}
\label{cap:progettazione-codifica}
%**************************************************************

\intro{Il seguente capitolo descrive gli strumenti e la progettazione con cui sono 
state implementate le integrazione con la \textit{web-app} \productName.}\\

%**************************************************************
\section{Tecnologie}
\label{sec:tecnologie-strumenti}

Di seguito viene data una panoramica delle tecnologie e strumenti utilizzati.

\subsection*{Java}
Java è un linguaggio di programmazione e una piattaforma di elaborazione 
rilasciato per la prima volta da Sun Microsystems nel 1995. Si è evoluto da umili
 origini per sostenere gran parte del mondo digitale di oggi, fornendo una
 piattaforma affidabile su cui sono costruiti molti servizi e applicazioni. 
 Anche i nuovi prodotti, innovativi nei servizi digitali progettati per il futuro,  
 continuano a fare affidamento su Java.
%TODO: aggiungere bibliografia, link: https://java.com/en/download/help/whatis_java.html
\subsection*{Spring}
Spring è un \textit{framework} open source per lo sviluppo di applicazioni su piattaforma Java.
A questo \textit{framework} sono associati tanti altri progetti, che hanno nomi composti come 
Spring Boot, Spring Data, Spring Batch, etc. Questi progetti sono stati ideati per
fornire funzionalità aggiuntive al \textit{framework}.
%TODO: aggiungere bibliografia, link: https://it.wikipedia.org/wiki/Spring_Framework

\subsection*{Typescript}
%**************************************************************
TypeScript è un linguaggio di programmazione sviluppato e gestito da Microsoft. 
È un \gls{superset} di JavaScript, che permette di aggiungere la tipizzazione 
statica opzionale al linguaggio. TypeScript è progettato per lo sviluppo di applicazioni 
di grandi dimensioni e per la \gls{transcompilazione} in JavaScript. Poiché TypeScript è un \gls{superset} di JavaScript, anche i programmi JavaScript esistenti sono validi programmi TypeScript.
%TODO: aggiungere bibliografia, link: https://en.wikipedia.org/wiki/TypeScript

\subsection*{Angular}
%**************************************************************
Angular è un \textit{framework} JavaScript per applicazioni \textit{web} dinamiche, utilizzato in particolare per la creazione di \gls{spa} e \textit{web-app}. Consente di utilizzare HTML come linguaggio template e di estenderne la sintassi per esprimere le componenti di un'applicazione in modo chiaro e succinto.
%TODO: aggiungere bibliografia, link: https://psicografici.com/angular-js/#:~:text=Angular%20%C3%A8%20un%20framework%20JavaScript,in%20modo%20chiaro%20e%20succinto.

\subsection*{Angular Material}
%**************************************************************
Angular Material è una libreria sviluppata da Google nel 2014 progettata per aiutare a sviluppare pagine \textit{web} in modo strutturato. \\
I suoi componenti aiutano a creare pagine \textit{web} e applicazioni \textit{web} attraenti, coerenti e funzionali.
%TODO: aggiungere bibliografia, link: https://psicografici.com/angular-js/#:~:text=Angular%20%C3%A8%20un%20framework%20JavaScript,in%20modo%20chiaro%20e%20succinto.

\subsection*{Node.js}
Node.js è una piattaforma di sviluppo open source per l'esecuzione di codice JavaScript lato \textit{server}. Node è utile per sviluppare applicazioni che richiedono una connessione permanente dal \textit{browser} al \textit{server} ed è spesso utilizzato per applicazioni in tempo reale come chat, feed di notizie e di notifiche.\\
Node.js è utilizzato da Angular per gestire le dipendenze, permettendo la dichiarazione di due insiemi di dipendenze: per gli sviluppatori e per far funzionare l'applicativo. In questo modo è possibile differenziare quali librerie si possono tralasciare in fase di \textit{deploy} dell'applicazione perché, ad esempio, necessarie solo per effettuare i test.
%TODO: aggiungere bibliografia, link: https://whatis.techtarget.com/definition/Nodejs#:~:text=James%20Denman-,Node.,feeds%20and%20web%20push%20notifications.

%**************************************************************
\section{Progettazione}
\label{sec:progettazione}

\subsection{Back-end}
\subsubsection{Architettura Spring}
\paragraph{Spring Boot}
\paragraph{Spring Core}
\paragraph{Spring Data JPA}

Descrizione dettagliata su come funziona Spring Boot, Spring Data JPA e Spring Core.

\subsubsection{Architettura Docker}
\subsubsection{Organizzazione del codice}
Struttura del progetto
\subsubsection{Progettazione delle API}


\subsection{Front-end}
\subsubsection{Architettura Angular}
Descrizione dettagliata su come funziona Angular
\subsubsection{Organizzazione del codice}

\subsubsection{Progettazione delle maschere}
Descrizione Figma e delle scelte realizzative del mockup realizzato.


\subsubsection{Namespace 1} %**************************
Descrizione namespace 1.

\begin{namespacedesc}
    \classdesc{Classe 1}{Descrizione classe 1}
    \classdesc{Classe 2}{Descrizione classe 2}
\end{namespacedesc}


%**************************************************************
\section{Design Pattern utilizzati}
\subsection{Microservizi}
\subsection{Api Gateway}
\subsection{Service Discovery}
\subsection{Service Registry}
\subsection{Dependency Injection}
\subsection{Singleton}
\subsection{Feature Service}
\subsection{Decorator}
\subsection{Lazy Loading}
\subsection{Observer}

%**************************************************************
\section{Codifica}
\subsection{Back-end}

\subsubsection{Gruppi service}
\subsubsection{Modifica dei servizi esistenti}
\subsubsection{Api Gateway ed Eureka Server}
\subsubsection{Docker}

\subsection{Front-end}
\subsubsection{Maschere}
\subsubsection{Componenti}