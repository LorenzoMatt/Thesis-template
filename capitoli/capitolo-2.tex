% !TEX encoding = UTF-8
% !TEX TS-program = pdflatex
% !TEX root = ../tesi.tex

%**************************************************************
\chapter{Descrizione dello stage}
\label{cap:descrizione-stage}
%**************************************************************

\intro{In questo capitolo è presente un'analisi preventiva dei rischi a cui si sarebbe potuto incombere durante lo \textit{stage}, la lista degli obiettivi da raggiungere e la pianificazione
delle ore di lavoro.}\\
%**************************************************************
\section{Analisi preventiva dei rischi}

Qui vengono analizzati i rischi emersi durante la fase di analisi iniziale a cui è possibile incombere.
Per ogni rischio individuato, si è proceduto ad elaborare un piano di contingenza per far fronte a tali rischi.

\begin{risk}{Inesperienza tecnologica}
    \riskdescription{le tecnologie da utilizzare sono nuove o esplorate parzialmente, il che può portare
     alla nascita di problemi operativi}
    \risksolution{le attività che richiedono maggior tempo oppure con un livello di capacità tecniche elevate
    saranno trattate per prime, in modo da migliorarle incrementalmente durante il periodo di stage}
    \label{risk:Inesperienza tecnologica} 
\end{risk}

\begin{risk}{Problematiche \textit{software} di supporto}
    \riskdescription{il \textit{computer} in cui si sviluppa il prodotto potrebbe guastarsi, e nel caso 
    accadesse potrebbe causare gravi ritardi}
    \risksolution{ad ogni \textit{commit} effettuato è necessario aggiornare la \textit{repository} remota.
    Inoltre deve essere possibile replicare velocemente l'ambiente di lavoro in un altro \textit{computer} 
    in modo da tornare operativi nel minor tempo possibile. Un modo per ripristinare velocemente le impostazioni
    di lavoro è la creazione di una cartella da versionare chiamata \texttt{.vscode} in cui inserire: 
    \begin{itemize}
        \item il file \texttt{settings.json}, in cui salvare le impostazioni dell'editor;
        \item il file \texttt{extensions.json}, in cui aggingere gli id di ogni estensione utilizzata in Visual Studio Code;
        \item Il file \texttt{launch.json}, che viene usato per configurare il \texttt{debugger} in Visual Studio Code. 
    \end{itemize}
    }\label{risk:Problematiche software di supporto}
\end{risk}

\begin{risk}{Tempistiche}
    \riskdescription{il tempo di apprendimento di nuove tecnologie potrebbe portare a ritardi sulle scadenze previste. 
    I ritardi verranno individuati nel caso in cui il lavoro da effettuare si scostasse dalla pianificazione presente su \gls{Trello}}
    \risksolution{appena si rilevano difficoltà o scostamenti rispetto al piano di lavoro, si dovrà avvisare tempestivamente il 
    \textit{tutor} aziendale, con cui ci si potrà confrontare, e solo in caso estremo rimandare le scadenze prefissate.}
    \label{risk:Tempistiche} 
\end{risk}

\begin{risk}{Impegni personali}
    \riskdescription{è possibile che vi siano degli impegni personali da adempiere e che di conseguenza abbia meno tempo
    da poter dedicare allo sviluppo del progetto.}
    \risksolution{gli incarichi con le relative scadenze sono stati predisposti nel rispetto degli eventuali impegni personali.
    In caso di imprevisti, bisognerà immediatamente contattare il \textit{tutor} aziendale.}
    \label{risk:Impegni personali} 
\end{risk}

\begin{risk}{Interpretazione errata o non sufficiente dei requisiti}
    \riskdescription{Dopo una prima analisi dei requisiti, è possibile che si noti la necessità di modificare o aggiungere
     nuovi requisiti in un secondo momento.}
    \risksolution{Due volte a settimana verrà fatto il punto della situazione con il \textit{tutor} aziendale. 
    Nel caso si notassero assenze nei requisiti, si procederà alla sua conseguante analisi e si deciderà come 
    procedere in maniera da limitare un eventuale rallentamento sulla di sviluppo.}
    \label{risk:Interpretazione errata o non sufficiente dei requisiti} 
\end{risk}
%**************************************************************
\section{Requisiti e obiettivi}
\subsection*{Notazione}
Si farà riferimento ai requisiti secondo le seguenti notazioni:
\begin{itemize}
	\item \textit{O} per i requisiti obbligatori, vincolanti in quanto obiettivo primario richiesto dal committente;
	\item \textit{D} per i requisiti desiderabili, non vincolanti o strettamente necessari,
		  ma dal riconoscibile valore aggiunto;
	\item \textit{F} per i requisiti facoltativi, rappresentanti valore aggiunto non strettamente 
		  competitivo.
\end{itemize}

Le sigle precedentemente indicate saranno seguite da una coppia sequenziale di numeri, identificativo del requisito.

\subsection*{Obiettivi fissati}
Si prevede lo svolgimento dei seguenti obiettivi:
\begin{itemize}
	\item Obbligatori
	\begin{itemize}
		\obiettiviObbligatori
	\end{itemize}
	
	\item Desiderabili 
	\begin{itemize}
		\obiettiviDesiderabili
	\end{itemize}
	
	\item Facoltativi
	\begin{itemize}
		\obiettiviFacoltativi
	\end{itemize} 
\end{itemize}


\section{Pianificazione del lavoro}

\subsection*{Pianificazione settimanale}
\prospettoSettimanale

\subsection*{Ripartizione ore}
La pianificazione, in termini di quantità di ore di lavoro, sarà così distribuita:
\begin{center}
    % Tabella da personalizzare in base alle ore delle attività

\begin{tabularx}{\textwidth}{|c|X|}
    \hline
    \textbf{Durata in ore} & \textbf{Descrizione dell'attività}                         \\\hline

    \textbf{160}           & \textbf{Formazione sulle tecnologie}                       \\ \hdashline

    \multirow{7}{0cm}                                                                   \\
    \textit{18}            &
    \textit{Studio Java Standard Edition e tool di sviluppo}                            \\
    \textit{18}            &
    \textit{Studio architettura a \glspl{microservizio}}                                \\
    \textit{4}             &
    \textit{Ripasso dei principi della buona programmazione (SOLID, CleanCode)}
    \\

    \textit{10}            &
    \textit{Studio teorico dell’architettura a \glspl{microservizio}: passaggio
    da monolite ad architetture a \glspl{microservizio}}                                \\
    \textit{15}            &
    \textit{Studio teorico dell’architettura a \glspl{microservizio}: Api
        Gateway, Service Discovery e Service Registry, Circuit Breaker e Saga Pattern}
    \\
    \textit{15}            &
    \textit{Studio Spring Core/Spring Boot}                                             \\

    \textit{20}            &
    \textit{Studio servizi REST e framework Spring Data REST}                           \\

    \textit{60}            &
    \textit{Studio ORM, in particolare il framework Spring Data JPA}                    \\

    \hline

    \textbf{40}            & \textbf{Definizione architettura di riferimento e relativa
    documentazione}                                                                     \\ \hdashline

    \multirow{3}{0cm}                                                                   \\
    \textit{14}            &
    \textit{Analisi del problema e del dominio applicativo}                             \\
    \textit{22}            &
    \textit{Progettazione della piattaforma e relativi test}                            \\
    \textit{4}             &
    \textit{Stesura documentazione relativa ad analisi e progettazione}                 \\
    \hline
    \textbf{80}            & \textbf{Implementazione del nuovo servizio}                \\
    \hline

    \textbf{40}            & \textbf{Collaudo Finale}                                   \\ \hdashline

    \multirow{4}{0cm}                                                                   \\
    \textit{30}            &
    \textit{Collaudo}                                                                   \\
    \textit{6}             &
    \textit{Stesura documentazione finale}                                              \\
    \textit{2}             &
    \textit{Incontro di presentazione della piattaforma con gli stakeholders}
    \\
    \textit{2}             &
    \textit{Live demo di tutto il lavoro di stage}                                      \\
    \hline

    \textbf{Totale ore}    & \multicolumn{1}{|c|}{\textbf{320}}                         \\\hline

\end{tabularx}
\end{center}
%**************************************************************

% TODO: vedere se aggiungere la sezione sotto
% \section{Processo sviluppo prodotto}