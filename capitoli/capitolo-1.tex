% !TEX encoding = UTF-8
% !TEX TS-program = pdflatex
% !TEX root = ../tesi.tex

%**************************************************************
\chapter{Introduzione}
\label{cap:introduzione}
%**************************************************************
\section{Convenzioni tipografiche}
\label{sec:convenzioni tipografiche}
Durante la stesura del documento sono state adottate le seguenti convenzioni tipografiche:
\begin{itemize}
	\item gli acronimi, le abbreviazioni e i termini ambigui o di uso non comune menzionati vengono definiti nel glossario, situato alla fine del presente documento;
	\item per la prima occorrenza dei termini riportati nel glossario viene utilizzata la seguente nomenclatura: {\color{RoyalBlue}{parola\glsfirstoccur}};
	\item i termini in lingua straniera o facenti parti del gergo tecnico sono evidenziati con il carattere \emph{corsivo}.
\end{itemize}
% % TODO: da rimuovere il testo scritto sotto

% \noindent Esempio di utilizzo di un termine nel glossario \\
% \gls{api}. \\

% \noindent Esempio di citazione in linea \\
% \cite{site:agile-manifesto}. \\

% \noindent Esempio di citazione nel pie' di pagina \\
% citazione\footcite{womak:lean-thinking} \\

%**************************************************************
\section{L'azienda}

\myCompany\xspace nasce a Napoli nel 2002 come \textit{software house} ed è rapidamente cresciuta nel
mercato dell'\gls{ICT}, tramutatasi in \textit{System Integrator} e conquistando significative fette di mercato nei settori 
\textit{mobile}, videosorveglianza e sicurezza delle infrastrutture informatiche aziendali. \\
Attualmente, \myCompany\xspace ha più di 150 clientidiretti e finali, con un organico aziendale
 di 200 dipendenti distribuiti tra le 5 sedi dislocate in tutta Italia.\\
\myCompany\xspace si pone come obiettivo principale quello di supportare il cliente nella realizzazione,
messa in opera e governance di soluzione IT, sia dal punto di vista tecnologico,
sia nel governo del cambiamento organizzativo.
\begin{figure}[!h] 
    \centering 
    \includegraphics[width=0.5\columnwidth]{logo_azienda.png} 
    \caption{Logo dell'azienda}
\end{figure}

%**************************************************************
\section{Lo stage proposto}
\epigraph{\label{eph: sport}\enquote*{Lo sport dà il meglio di sé quando ci unisce.}}{Frank Deford}
\noindent \productName permette agli utenti di condividere le proprie \gls{will}, 
e gli utenti che hanno intenzione di unirsi in questa attività vi possono partecipare.\\
Dal momento che allo stato dell'arte attuale non esiste ancora la suddivisione delle \gls{will}
per gruppi, lo stage proposto da \myCompany consiste nell'integrare alla piattaforma già esistente
la suddivisione delle visualizzazioni delle \gls{will} solo agli utenti che 
appartengono agli stessi gruppi.\\
Gli obbiettivi da raggiungere nel corso dello stage sono principalmente due:
\begin{itemize}
    \item sviluppo di un microservizio utilizzando il \gls{framework} \gls{Spring} Java per la creazione dei gruppi;
    \item modifica dei microservizi esistenti affinchè permettano la visualizzazione solo delle \gls{will} di utenti appartenenti allo stesso gruppo.
\end{itemize}
Non è richiesta la modifica del \textit{front-end} in modo da adeguarlo alle nuove funzionalità del \textit{back-end}, a meno che 
non rimanga tempo da investire su questa attività.


%**************************************************************
\section{Organizzazione del testo}

\begin{description}
    \item[{\hyperref[cap:processi-metodologie]{Il secondo capitolo}}] descrive ...
    
    \item[{\hyperref[cap:descrizione-stage]{Il terzo capitolo}}] approfondisce ...
    
    \item[{\hyperref[cap:analisi-requisiti]{Il quarto capitolo}}] approfondisce ...
    
    \item[{\hyperref[cap:progettazione-codifica]{Il quinto capitolo}}] approfondisce ...
    
    \item[{\hyperref[cap:verifica-validazione]{Il sesto capitolo}}] approfondisce ...
    
    \item[{\hyperref[cap:conclusioni]{Nel settimo capitolo}}] descrive ...
\end{description}
