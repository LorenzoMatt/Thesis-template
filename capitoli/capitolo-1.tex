% !TEX encoding = UTF-8
% !TEX TS-program = pdflatex
% !TEX root = ../tesi.tex

%**************************************************************
\chapter{Introduzione}
\label{cap:introduzione}

%**************************************************************
\section{L'azienda}

\myCompany\xspace nasce a Napoli nel 2002 come \textit{software house} ed è
rapidamente cresciuta nel
mercato dell'\gls{ICT}, tramutatasi in \textit{System Integrator} e
conquistando significative fette di mercato nei settori
\textit{mobile}, videosorveglianza e sicurezza delle infrastrutture
informatiche aziendali. \\
Attualmente, \myCompany ha più di 150 clienti diretti e finali, con un organico
aziendale
di 200 dipendenti distribuiti tra le 5 sedi dislocate in tutta Italia.\\
\myCompany\xspace si pone come obiettivo principale quello di supportare il
cliente nella realizzazione,
messa in opera e \textit{governance} di soluzione IT, sia dal punto di vista
tecnologico,
sia nel governo del cambiamento organizzativo.
\begin{figure}[!h]
      \centering
      \includegraphics[width=0.5\columnwidth]{logo_azienda.png}
      \caption{Logo dell'azienda}
\end{figure}

%**************************************************************
\section{Lo stage proposto}
\epigraph{\label{eph: sport}\enquote*{Lo sport dà il meglio di sé quando ci
            unisce.}}{Frank Deford}
\noindent La funzionalità principale di \productName è quella di permettere
agli utenti di condividere le proprie
\gls{will} e partecipare a queste attività sportive. Tuttavia, allo stato
dell'arte attuale, non esiste ancora la suddivisione
delle \gls{will}
per gruppi, quindi lo \textit{stage} proposto da \myCompany consiste
nell'integrare
alla piattaforma già esistente
la visualizzazione delle \gls{will} agli utenti che appartengono agli stessi
gruppi.\\ \\
L'azienda possiede già i dati per la gestione degli utenti e delle
\gls{will}, le quali vanno però associate ai gruppi. In particolare, è
necessario
tenere
traccia delle \gls{will} e degli utenti appartenenti ad ogni gruppo, risultato
che si ottiene attraverso
la creazione di tabelle che congiungono gli
utenti e le \gls{will} con la tabella dei gruppi.\\ \\
Le implementazioni alla \textit{web app} permettono di visualizzare le
\gls{will} appartenenti
ad ogni gruppo. Inoltre, è ora possibile per l'utente visualizzare i gruppi a cui partecipa, quelli a cui può unirsi e quelli da lui creati e
modificare ed eliminare questi ultimi. \\ \\
È stata prestata attenzione alla consistenza e alla coerenza dei dati in caso
di eliminazione di un gruppo, di un utente o di una \gls{will} dal
\textit{database}.
Infatti, nel caso di eliminazione occorre
aggiornare le voci nelle tabelle di congiunzione corrispondenti.

\section{Strumenti utilizzati}
Nella prima fase del progetto sono state apportate modifiche al \textit{back
      end}. Gli strumenti utilizzati in questa fase sono:
\begin{description}
      \item[Visual Studio Code] Editor di codice con
            funzionalità di \textit{debugging}, controllo di versione con Git
            integrato,
            \textit{syntax highlighting}, \textit{intelliSense},
            \textit{snippet} e \textit{refactoring} del codice.\\
            Il punto di forza di Visual Studio Code sono le estensioni, grazie
            alle quali è possibile ampliare notevolmente le funzionalità del
            programma.\\
            Le estensioni utilizzate nel corso di questa fase sono:
            \begin{itemize}
                  \item \textbf{GitLens}: amplia le funzionalità di Git
                        integrate in
                        Visual Studio Code;
                  \item \textbf{Spring Boot Extension Pack}: raccolta di
                        estensioni per lo sviluppo con Spring Boot;

                  \item \textbf{Docker Extension Pack}: raccolta di estensioni
                        per lo
                        gestione dei \gls{container} Docker, Docker images,
                        Dockerfile e
                        file
                        Docker-compose;

                  \item \textbf{Extension Pack for Java}: raccolta di
                        estensioni popolari
                        che possono aiutare a scrivere, testare e fare il
                        \textit{debugging} di applicazioni Java.
            \end{itemize}
      \item[Postman]
            Strumento che permette di eseguire richieste HTTP ad un
            \textit{server} di \textit{back end}. È stato utilizzato per
            testare le richiesta HTTP dei servizi Rest sviluppati.
      \item[DbVisualizer] Strumento \textit{multi-database} intuitivo e ricco
            di
            funzionalità per sviluppatori, analisti e amministratori di
            \textit{database},
            che fornisce un'unica interfaccia su un'ampia gamma di
            sistemi
            operativi. \\
            DbVisualizer ha un'interfaccia chiara, facile da usare ed è uno
            strumento
            che
            funziona su tutti i principali sistemi operativi, supportando molte
            varietà di
            \textit{database}. \\
            È stato utilizzato per visualizzare il popolamento delle tabelle
            di un \textit{database} PostgreSQL, in particolare dopo aver
            effettuato delle modifiche attraverso le richieste HTTP.
\end{description}

\noindent Nella seconda fase dello sviluppo, in cui sono state apportate
modifiche al
\textit{front end}, sono state progettate le interfacce delle pagine
\textit{web} per la visualizzazione, creazione e modifica di un gruppo da
implementare con \textbf{Figma}, un \textit{tool} per la progettazione di
interfacce che si rivolge principalmente ai \textit{web designer} che hanno
bisogno di un
\textit{software} studiato appositamente per realizzare il \textit{design} di
siti \textit{web} e applicazioni.\\

\noindent Per quanto riguarda la codifica, come nella prima fase, è stato
utilizzato Visual Studio
Code come \textit{editor} di codice, con l'aiuto di nuove estensioni per lo
sviluppo di applicazioni \textit{web}, ovvero:
\begin{itemize}
      \item \textbf{Angular Extension Pack}: raccolta di
            estensioni per lo sviluppo con Angular;
      \item \textbf{W3C Web Validator}: controlla la validità del
            \textit{markup} dei documenti HTML e CSS;
      \item \textbf{Web Accessibility}: verifica l'accessibilità
            dei documenti HTML, evidenziando gli elementi che si
            potrebbe prendere in considerazione di cambiare e dando
            suggerimenti su come potrebbero essere modificati.
\end{itemize}

\section{Prodotto ottenuto}
Al termine dello \textit{stage} le integrazioni delle funzionalità con la
\textit{web app} sono state realizzate con successo. Tutte le chiamate alle
\gls{API}
sono state testate e integrate anche nel \textit{front end}.
L'integrazione delle nuove funzionalità permettono alla \textit{web app} di:
\begin{itemize}
      \item  visualizzare le \gls{will} appartenenti agli utenti che
            partecipano
            agli stessi gruppi;
      \item visualizzare le \gls{will} con visibilità globale (quindi che non
            sono visibili solo agli utenti che partecipano agli stessi gruppi);
      \item visualizzare i gruppi a cui partecipa un utente;
      \item visualizzare e modificare i gruppi creati da un utente;
      \item visualizzare e cercare i gruppi;
      \item creare nuovi gruppi.
\end{itemize}
\section{Organizzazione del testo}

\begin{description}
      \item[{\hyperref[cap:descrizione-stage]{Il secondo capitolo}}] descrive
      l'analisi preventiva dei rischi, gli obiettivi dello \textit{stage} e
      la pianificazione delle ore di lavoro.
      \item[{\hyperref[cap:analisi-requisiti]{Il terzo capitolo}}]
      approfondisce
      l'analisi dei requisiti del prodotto.

      \item[{\hyperref[cap:progettazione-codifica]{Il quarto capitolo}}]
      approfondisce la fase di progettazione e codifica.

      \item[{\hyperref[cap:verifica-validazione]{Il quinto capitolo}}]
      approfondisce l'accessibilità e la fase di verifica e validazione.

      \item[{\hyperref[cap:conclusioni]{Il sesto capitolo}}] contiene
      un’analisi
      del lavoro svolto e le conclusioni tratte.
\end{description}

%**************************************************************
\section{Convenzioni tipografiche}
\label{sec:convenzioni tipografiche}
Durante la stesura del documento sono state adottate le seguenti convenzioni
tipografiche:
\begin{itemize}
      \item gli acronimi, le abbreviazioni e i termini ambigui o di uso non
            comune menzionati vengono definiti nel glossario, situato alla fine
            del
            presente documento;
      \item per la prima occorrenza dei termini riportati nel glossario viene
            utilizzata la seguente nomenclatura:
            {\color{RoyalBlue}{parola\glsfirstoccur}};
      \item i termini in lingua straniera o facenti parti del gergo tecnico
            sono evidenziati con il carattere \emph{corsivo}.
\end{itemize}