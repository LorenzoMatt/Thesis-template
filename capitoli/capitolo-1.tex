% !TEX encoding = UTF-8
% !TEX TS-program = pdflatex
% !TEX root = ../tesi.tex

%**************************************************************
\chapter{Introduzione}
\label{cap:introduzione}
%**************************************************************
\section{Convenzioni tipografiche}
\label{sec:convenzioni tipografiche}
Durante la stesura del documento sono state adottate le seguenti convenzioni tipografiche:
\begin{itemize}
	\item gli acronimi, le abbreviazioni e i termini ambigui o di uso non comune menzionati vengono definiti nel glossario, situato alla fine del presente documento;
	\item per la prima occorrenza dei termini riportati nel glossario viene utilizzata la seguente nomenclatura: {\color{RoyalBlue}{parola\glsfirstoccur}};
	\item i termini in lingua straniera o facenti parti del gergo tecnico sono evidenziati con il carattere \emph{corsivo}.
\end{itemize}
% % TODO: da rimuovere il testo scritto sotto

% \noindent Esempio di utilizzo di un termine nel glossario \\
% \gls{API}. \\

% \noindent Esempio di citazione in linea \\
% \cite{site:agile-manifesto}. \\

% \noindent Esempio di citazione nel pie' di pagina \\
% citazione\footcite{womak:lean-thinking} \\

%**************************************************************
\section{L'azienda}

\myCompany\xspace nasce a Napoli nel 2002 come \textit{software house} ed è rapidamente cresciuta nel
mercato dell'\gls{ICT}, tramutatasi in \textit{System Integrator} e conquistando significative fette di mercato nei settori 
\textit{mobile}, videosorveglianza e sicurezza delle infrastrutture informatiche aziendali. \\
Attualmente, \myCompany\xspace ha più di 150 clientidiretti e finali, con un organico aziendale
 di 200 dipendenti distribuiti tra le 5 sedi dislocate in tutta Italia.\\
\myCompany\xspace si pone come obiettivo principale quello di supportare il cliente nella realizzazione,
messa in opera e governance di soluzione IT, sia dal punto di vista tecnologico,
sia nel governo del cambiamento organizzativo.
\begin{figure}[!h] 
    \centering 
    \includegraphics[width=0.5\columnwidth]{logo_azienda.png} 
    \caption{Logo dell'azienda}
\end{figure}

%**************************************************************
\section{Lo stage proposto}
\epigraph{\label{eph: sport}\enquote*{Lo sport dà il meglio di sé quando ci unisce.}}{Frank Deford}
\noindent \productName permette agli utenti di condividere le proprie \gls{will}, 
e gli utenti che hanno intenzione di unirsi in questa attività vi possono partecipare.\\
Dal momento che allo stato dell'arte attuale non esiste ancora la suddivisione delle \gls{will}
per gruppi, lo \textit{stage} proposto da \myCompany consiste nell'integrare alla piattaforma già esistente
la suddivisione delle visualizzazioni delle \gls{will} solo agli utenti che 
appartengono agli stessi gruppi.\\
Gli obbiettivi da raggiungere nel corso dello \textit{stage} sono principalmente due:
\begin{itemize}
    \item sviluppo di un \gls{microservizio} utilizzando il \gls{framework} \gls{Spring} Java per la creazione dei gruppi;
    \item modifica dei \glspl{microservizio} esistenti affinchè permettano la visualizzazione solo delle \gls{will} di utenti appartenenti allo stesso gruppo.
\end{itemize}
Non è richiesta la modifica del \textit{frontend} in modo da adeguarlo alle nuove funzionalità del \textit{backend}, a meno che 
non rimanga tempo da investire su questa attività.

\section{Strumenti utilizzati}
\subsection{Visual Studio Code}
Visual Studio Code è un editor di codice sorgente sviluppato da Microsoft per Windows, Linux e macOS. 
Include il supporto per debugging, un controllo per Git integrato, Syntax highlighting, IntelliSense, 
Snippet e refactoring del codice.\\
Punto di forza di Visual Studio Code sono le estensioni grazie alle quali è possibile ampliare 
notevolmente le funzionalità del programma.\\
Le estensioni utilizzate nel corso dello \textit{stage} sono:
    \begin{itemize}
        \item \textbf{GitLens}: permette di ampliare le funzionalità di Git integrate in Visual Studio Code;
        \item \textbf{Spring Boot Extension Pack}: raccolta di estensioni per lo sviluppo con Spring Boot Application;
        \item \textbf{W3C Web Validator}: permette di controllare la validità del \textit{markup} dei documenti html e css;
        \item \textbf{Web Accessibility}: permette di verificare l'accessibilità dei documenti html, evidenziando gli 
        elementi che si potrebbe prendere in conisderazione di cambiare e dando suggerimenti su come potrebbe modificato;
        \item \textbf{Docker Extension Pack}: raccolta di estensioni per lo gestione dei \textit{container} Docker, Docker images, Dockerfile e file Docker-compose;
        \item \textbf{Angular Extension Pack}: raccolta di estensioni per lo sviluppo con Angular;
        \item \textbf{Extension Pack for Java}: raccolta di estensioni popolari che possono aiutare a scrivere, testare e fare il
        \textit{debugging} di applicazioni Java.
    \end{itemize}
\subsection{Figma}
Figma è un \textit{tool} per la progettazione di interfacce, che si rivolge principalmente ai \textit{web designer} che hanno bisogno di un software studiato 
appositamente per realizzare il \textit{design} di siti web e applicazioni.\\
Nel contesto dello \textit{stage} è stato utilizzato per la realizzazione delle pagine web per la visualizzazione, creazione e modifica di un gruppo. 
\subsection{Git}
\textit{Software} di versionamento utile a tracciare modifiche e cambiamenti di insiemi di file.
\subsection{Postman}
Si tratta di uno strumento che permette di eseguire richieste HTTP ad un \textit{server} di \textit{backend}. Quando si lavora con un altro sviluppatore 
\textit{backend} è possibile condividere le \gls{API}, ma la sua vera forza è quella di farci sapere tutto di una richiesta HTTP.
\subsection{DbVisualizer}
DbVisualizer è uno strumento multi-database intuitivo e ricco di funzionalità per sviluppatori, analisti e amministratori di database, che fornisce 
un'unica, potente interfaccia su un'ampia gamma di sistemi operativi. Grazie alla sua interfaccia chiara e facile da usare, DbVisualizer si è dimostrato 
uno strumento di database molto conveniente, che funziona su tutti i principali sistemi operativi e supporta molte varietà di database. 


\section{Prodotto ottenuto}
Al termine dello \textit{stage} le integrazioni delle funzionalità con la \textit{web-app} sono state realizzata con successo. Tutte le chiamate alle \gls{API} 
sono state testate e sono state integrate anche nel \textit{frontend}. 
L'integrazione delle nuove funzionalità permettono alla \textit{web-app} di: 
\begin{itemize}
    \item  visualizzare le \gls{will} appartenenti agli utenti che partecipano agli stessi 
    gruppi;
    \item visualizzare le \gls{will} con visibilità globale (quindi che non sono visibili solo agli utenti che partecipano agli stessi gruppi);
    \item visualizzare i gruppi a cui partecipa un utente;
    \item visualizzare e modificare i gruppi creati da un utente;
    \item visualizzare e cercare i gruppi;
    \item creare nuovi gruppi.
\end{itemize}
% TODO: da vedere dove inserire la prossima sezione
\pagebreak
\section{Accessibilità}
Durante la realizzazione del sito è stata resa la navigazione più efficace attraverso l'uso di accesskey.\\
Le immagini sono tutte marcate con gli appositi \textit{tag alt}, che sono stati lasciati vuoti nel caso servissero solo per il \textit{layout}.\\
È presente una barra di navigazione che aiuta a navigare nel sito.\\ 
Ogni \textit{link} è stato reso distinguibile da ogni altro elemento tramite appositi CSS, \textit{hover} e \textit{visited}, in modo da aiutare l'utente ad orientarsi. \\
Inoltre, per evitare \textit{link} circolari, nella barra di navigazione le pagine non contengono \textit{link} che navigano alla pagina corrente.\\
I form contengono dei tag label per ogni input.\\ 
Non sono stati aggiunti \textit{tag optgroup} o \textit{fieldset} in quanto sono utili nel caso di \textit{form} molto grandi, ma essendo presenti solo \textit{form} di piccole dimensioni è stato ritenuto non necessario.\\ 
Sono presenti gli attributi \textit{accesskey} con chiave uguale alla prima lettera della parola del \textit{tag label} associato per migliorare l'accessibilità alle \textit{form} da tastiera senza l'uso del \textit{mouse}. 
Sono presenti degli aiuti contestuali che mostrano gli errori nel caso fossero presenti. Non sono stati aggiunti \textit{tabindex} nei \textit{form} dato che l'ordine di tabulazione è già corretto.\\
Sono stato evitati i tag ed attributi deprecati.\\
È presente del testo nascosto utile agli utenti con disabilità visive, come il \textit{link} con la funzione  di saltare al contenuto, ovvero di permettere di non far leggere allo \textit{screen reader} la barra di navigazione, passando direttamente al contenuto, ed è presente all'inizio della navigazione
 per segnalare che le scorciatoie da tastiera sono attive.\\
Inoltre per migliorare la navigazione dello \textit{scroll}, viene mostrato un pulsante in basso
a destra dello schermo che, se un utente lo clicca, viene effettuato uno \textit{scroll} verso l'alto fino all'inizio della pagina.\\
Il pulsante è un \textit{link} con testo nascosto e come immagine di \textit{background} una freccia, in modo che lo \textit{screen reader} riesca comunque a leggere il testo.
Le parole in lingua straniera sono state racchiuse dentro un \textit{tag} con l'attributo \textit{lang}, in modo da permettere allo \textit{screen reader} di leggere la parola correttamente.
\section{Organizzazione del testo}

\begin{description}    
    \item[{\hyperref[cap:descrizione-stage]{Il secondo capitolo}}] descrive l'analisi preventiva dei rischi, gli obiettivi dello \textit{stage} e
    la pianificazione delle ore di lavoro.
    \item[{\hyperref[cap:analisi-requisiti]{Il terzo capitolo}}] approfondisce l'analisi dei requisiti del prodotto.
    
    \item[{\hyperref[cap:progettazione-codifica]{Il quarto capitolo}}] approfondisce la fase di progettazione e codifica.
    
    \item[{\hyperref[cap:verifica-validazione]{Il quinto capitolo}}] approfondisce l'accessibilità e la fase di verifica e validazione.
    
    \item[{\hyperref[cap:conclusioni]{Il sesto capitolo}}] contiene un’analisi del lavoro svolto e le conclusioni tratte.
\end{description}
