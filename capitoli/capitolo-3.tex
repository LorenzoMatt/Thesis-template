% !TEX encoding = UTF-8
% !TEX TS-program = pdflatex
% !TEX root = ../tesi.tex

%**************************************************************
\chapter{Analisi dei requisiti}
\label{cap:analisi-requisiti}
%**************************************************************

\intro{Il presente capitolo descrive in maniera dettagliata requisiti e casi d'uso individuati durante la fase di analisi del progetto di \textit{stage}.}\\

\section{Casi d'uso}
\subsection*{Attori pricipali}
Nella fase di analisi del progetto di \textit{stage} è emersa la presenza di un solo attore, ovvero l'\textbf{utente autenticato}, 
attore che rappresenta un utente che ha effettuato l'autenticazione
all'interno dell'applicazione \textit{web}. Ha quindi la possibilità di vedere tutte le informazioni sui gruppi che sarebbero altrimenti inaccessibili. 

\subsection*{Elenco dei casi d'uso}
Per lo studio dei casi di utilizzo del prodotto sono stati creati dei diagrammi dei casi d'uso.
I diagrammi dei casi d'uso (in inglese \emph{Use Case Diagram}) sono diagrammi di tipo \gls{uml} dedicati alla descrizione delle funzioni o servizi offerti da un sistema, così come sono percepiti e utilizzati dagli attori che interagiscono col sistema stesso.\\
I casi d'uso riportati in seguito descrivono solo le funzionalità che dovranno essere implementate, senza quindi descrivere quelle già presenti nella \textit{web app}. 
\begin{figure}[H] 
    \centering 
    \includegraphics[width=1\columnwidth]{usecase/diagramma generale dei casi d'uso.png} 
    \caption{Diagramma generale dei casi d'uso}
\end{figure}

\begin{usecase}{Visualizzazione lista dei gruppi}
    \label{uc:scenario-visualizzazione-lista-gruppi}
    \usecaseactors{Utente autenticato}
    \usecasepre{L'utente ha aperto la \textit{web app} ed è autenticato}
    \usecasedesc{L'utente vuole visualizzare la lista dei gruppi}
    \usecasepost{L'utente ha visualizzato la lista dei gruppi}
    \usecasescenarioprincipale{
        \begin{enumerate}[nolistsep]
            \item L'utente visualizza la lista dei gruppi.
        \end{enumerate}
    }
    
\end{usecase}
\newpage

\begin{figure}[H] 
    \centering 
    \includegraphics[width=1.2\columnwidth]{usecase/sottocasi UC1.png} 
    \caption{UC1: Vis. lista dei gruppi}
\end{figure}

\begin{subusecase}{Visualizzazione singolo gruppo in lista}
    \label{sub:visualizzazione-singolo-gruppo}
    \usecaseactors{Utente autenticato}
    \usecasepre{L'utente ha aperto la \textit{web app}, è autenticato e sta visualizzando la lista dei gruppi}
    \usecasedesc{L'utente vuole visualizzare un singolo gruppo nella lista dei gruppi}
    \usecasepost{L'utente ha visualizzato un singolo gruppo nella lista dei gruppi}
    \usecasescenarioprincipale{
        \begin{enumerate}[nolistsep]
            \item L'utente visualizza la pagina contenente la lista dei gruppi;
            \item l'utente visualizza un singolo gruppo nella lista dei gruppi.
        \end{enumerate}
    }
\end{subusecase}

\begin{figure}[H] 
    \centering 
    \includegraphics[width=1.2\columnwidth]{usecase/sottocasi UC1.1.png} 
    \caption{UC1.1: Vis. singolo gruppo in lista}
\end{figure}

\newpage

\begin{subsubusecase}{Visualizzazione nome del gruppo}
    \label{subsub:visualizzazione-nome-gruppo}
    \usecaseactors{Utente autenticato}
    \usecasepre{L'utente ha aperto la \textit{web app}, è autenticato e sta visualizzando un gruppo dalla lista dei gruppi}
    \usecasedesc{L'utente vuole visualizzare il nome di un gruppo dalla lista dei gruppi}
    \usecasepost{L'utente ha visualizzato il nome di un gruppo dalla lista dei gruppi}
    \usecasescenarioprincipale{
        \begin{enumerate}[nolistsep]
            \item L'utente visualizza la pagina contenente la lista dei gruppi;
            \item l'utente visualizza il nome di un gruppo dalla lista dei gruppi.
        \end{enumerate}
    }
\end{subsubusecase}


\begin{subsubusecase}{Visualizzazione descrizione del gruppo}
    \label{subsub:visualizzazione-descrizione-gruppo}
    \usecaseactors{Utente autenticato}
    \usecasepre{L'utente ha aperto la \textit{web app}, è autenticato e sta visualizzando un gruppo dalla lista dei gruppi}
    \usecasedesc{L'utente vuole visualizzare la descrizione di gruppo della lista dei gruppi}
    \usecasepost{L'utente ha visualizzato la descrizione di un gruppo dalla lista dei gruppi}
    \usecasescenarioprincipale{
        \begin{enumerate}[nolistsep]
            \item L'utente visualizza la pagina contenente la lista dei gruppi;
            \item l'utente visualizza la descrizione di un gruppo dalla lista dei gruppi.
        \end{enumerate}
    }
\end{subsubusecase}


\begin{usecase}{Visualizzazione lista dei gruppi creati dall'utente}
    \label{uc:scenario-visualizzazione-lista-gruppi-creati}
    \usecaseactors{Utente autenticato}
    \usecasepre{L'utente ha aperto la \textit{web app} ed è autenticato}
    \usecasedesc{L'utente vuole visualizzare la lista dei gruppi creati da lui}
    \usecasepost{L'utente ha visualizzato la lista dei gruppi creati da lui}
    \usecasescenarioprincipale{
        \begin{enumerate}[nolistsep]
            \item L'utente visualizza la lista dei gruppi creati da lui.
        \end{enumerate}
    }
\end{usecase}

\begin{usecase}{Visualizzazione lista dei gruppi a cui partecipa un utente}
    \label{uc:scenario-visualizzazione-lista-gruppi-partecipa}
    \usecaseactors{Utente autenticato}
    \usecasepre{L'utente ha aperto la \textit{web app} ed è autenticato}
    \usecasedesc{L'utente vuole visualizzare la lista dei gruppi a cui partecipa}
    \usecasepost{L'utente ha visualizzato la lista dei gruppi a cui partecipa}
    \usecasescenarioprincipale{
        \begin{enumerate}[nolistsep]
            \item L'utente visualizza la lista dei gruppi a cui partecipa.
        \end{enumerate}
    }
\end{usecase}


\begin{usecase}{Visualizzazione lista dei gruppi creati da altri utenti}
    \label{uc:scenario-visualizzazione-lista-gruppi-altri}
    \usecaseactors{Utente autenticato}
    \usecasepre{L'utente ha aperto la \textit{web app} ed è autenticato}
    \usecasedesc{L'utente vuole visualizzare la lista dei gruppi creati da altri utenti}
    \usecasepost{L'utente ha visualizzato la lista dei gruppi creati da altri utenti}
    \usecasescenarioprincipale{
        \begin{enumerate}[nolistsep]
            \item L'utente visualizza la lista dei gruppi creati da altri utenti.
        \end{enumerate}
    }
\end{usecase}


\begin{usecase}{Visualizzazione dettagli di un gruppo}
    \label{uc:scenario-visualizzazione-dettaglio-gruppo}
    \usecaseactors{Utente autenticato}
    \usecasepre{L'utente ha aperto la \textit{web app}, è autenticato e sta visualizzando la lista dei gruppi}
    \usecasedesc{L'utente vuole visualizzare i dettagli di un gruppo}
    \usecasepost{L'utente ha visualizzato i dettagli di un gruppo}
    \usecasescenarioprincipale{
        \begin{enumerate}[nolistsep]
            \item L'utente clicca su un gruppo sul quale vuole ottenere i dettagli; 
            \item l'utente visualizza i dettagli di un gruppo.
        \end{enumerate}
    }
\end{usecase}
\begin{figure}[H] 
    \centering 
    \includegraphics[width=0.9\columnwidth]{usecase/sottocasi UC5.png} 
    \caption{UC5: Vis. dettagli di un gruppo}
\end{figure}
\newpage


\begin{subusecase}{Visualizzazione nome del gruppo}
    \label{sub:visualizzazione-nome-gruppo}
    \usecaseactors{Utente autenticato}
    \usecasepre{L'utente ha aperto la \textit{web app}, è autenticato e sta visualizzando i dettagli di un gruppo}
    \usecasedesc{L'utente vuole visualizzare il nome di un gruppo}
    \usecasepost{L'utente ha visualizzato il nome di un gruppo}
    \usecasescenarioprincipale{
        \begin{enumerate}[nolistsep]
            \item L'utente visualizza il nome di un gruppo.
        \end{enumerate}
    }
\end{subusecase}

\begin{subusecase}{Visualizzazione immagine del gruppo}
    \label{sub:visualizzazione-immagine-gruppo}
    \usecaseactors{Utente autenticato}
    \usecasepre{L'utente ha aperto la \textit{web app}, è autenticato e sta visualizzando i dettagli di un gruppo}
    \usecasedesc{L'utente vuole visualizzare l'immagine di un gruppo}
    \usecasepost{L'utente ha visualizzato l'immagine di un gruppo}
    \usecasescenarioprincipale{
        \begin{enumerate}[nolistsep]
            \item L'utente visualizza l'immagine di un gruppo.
        \end{enumerate}
    }
\end{subusecase}

\begin{subusecase}{Visualizzazione lista dei partecipanti di un gruppo}
    \label{sub:visualizzazione-utenti-gruppo}
    \usecaseactors{Utente autenticato}
    \usecasepre{L'utente ha aperto la \textit{web app}, è autenticato e sta visualizzando i dettagli di un gruppo}
    \usecasedesc{L'utente vuole visualizzare la lista dei partecipanti ad un gruppo}
    \usecasepost{L'utente ha visualizzato la lista dei partecipanti ad un gruppo}
    \usecasescenarioprincipale{
        \begin{enumerate}[nolistsep]
            \item L'utente visualizza la lista dei partecipanti ad un gruppo.
        \end{enumerate}
    }
\end{subusecase}

\begin{figure}[H] 
    \centering 
    \includegraphics[width=1\columnwidth]{usecase/sottocasi UC5.3.png} 
    \caption{UC5.3: Vis. lista dei partecipanti di un gruppo}
\end{figure}

\begin{subsubusecase}{Visualizzazione singolo partecipante in lista}
    \label{subsub:visualizzazione-partecipanti-gruppo}
    \usecaseactors{Utente autenticato}
    \usecasepre{L'utente ha aperto la \textit{web app}, è autenticato e sta visualizzando la lista dei partecipanti ad un gruppo}
    \usecasedesc{L'utente vuole visualizzare un partecipante dalla lista dei partecipanti ad un gruppo}
    \usecasepost{L'utente ha visualizzato un partecipante dalla lista dei partecipanti ad un gruppo}
    \usecasescenarioprincipale{
        \begin{enumerate}[nolistsep]
            \item L'utente visualizza un partecipante dalla lista dei partecipanti ad un gruppo.
        \end{enumerate}
    }
\end{subsubusecase}

\begin{figure}[H] 
    \centering 
    \includegraphics[width=1\columnwidth]{usecase/sottocasi UC5.3.1.png} 
    \caption{UC5.3.1: Vis. singolo partecipante in lista}
\end{figure}

\begin{subsubsubusecase}{Visualizzazione singolo partecipante in lista}
    \label{subsubsub:visualizzazione-singolo-partecipante-gruppo}
    \usecaseactors{Utente autenticato}
    \usecasepre{L'utente ha aperto la \textit{web app}, è autenticato e sta visualizzando un partecipante dalla lista dei partecipanti}
    \usecasedesc{L'utente vuole visualizzare nome e cognome di un partecipante dalla lista dei partecipanti}
    \usecasepost{L'utente ha visualizzato nome e cognome di un partecipante dalla lista dei partecipanti}
    \usecasescenarioprincipale{
        \begin{enumerate}[nolistsep]
            \item L'utente visualizza nome e cognome di un partecipante dalla lista dei partecipanti.
        \end{enumerate}
    }
\end{subsubsubusecase}

\begin{subusecase}{Visualizzazione descrizione del gruppo}
    \label{sub:visualizzazione-descrizione-gruppo}
    \usecaseactors{Utente autenticato}
    \usecasepre{L'utente ha aperto la \textit{web app}, è autenticato e sta visualizzando i dettagli di un gruppo}
    \usecasedesc{L'utente vuole visualizzare la descrizione di un gruppo}
    \usecasepost{L'utente ha visualizzato la descrizione di un gruppo}
    \usecasescenarioprincipale{
        \begin{enumerate}[nolistsep]
            \item L'utente visualizza la descrizione di un gruppo.
        \end{enumerate}
    }
\end{subusecase}

\begin{subusecase}{Visualizzazione obbiettivi del gruppo}
    \label{sub:visualizzazione-obbiettivi-gruppo}
    \usecaseactors{Utente autenticato}
    \usecasepre{L'utente ha aperto la \textit{web app}, è autenticato e sta visualizzando i dettagli di un gruppo}
    \usecasedesc{L'utente vuole visualizzare gli obbiettivi di un gruppo}
    \usecasepost{L'utente ha visualizzato gli obbiettivi di un gruppo}
    \usecasescenarioprincipale{
        \begin{enumerate}[nolistsep]
            \item L'utente visualizza gli obbiettivi di un gruppo.
        \end{enumerate}
    }
\end{subusecase}


\begin{subusecase}{Visualizzazione località del gruppo}
    \label{sub:visualizzazione-località-gruppo}
    \usecaseactors{Utente autenticato}
    \usecasepre{L'utente ha aperto la \textit{web app}, è autenticato e sta visualizzando i dettagli di un gruppo}
    \usecasedesc{L'utente vuole visualizzare la località di un gruppo}
    \usecasepost{L'utente ha visualizzato la località di un gruppo}
    \usecasescenarioprincipale{
        \begin{enumerate}[nolistsep]
            \item L'utente visualizza la località di un gruppo.
        \end{enumerate}
    }
\end{subusecase}

\begin{usecase}{Modifica di un gruppo}
    \label{uc:scenario-modifica-gruppo}
    \usecaseactors{Utente autenticato}
    \usecasepre{L'utente ha aperto la \textit{web app}, è autenticato e sta visualizzando la lista dei gruppi creati da lui}
    \usecasedesc{L'utente vuole modificare i dettagli di un gruppo}
    \usecasepost{L'utente ha modificato i dettagli di un gruppo}
    \usecasescenarioprincipale{
        \begin{enumerate}[nolistsep]
            \item L'utente modifica i dettagli di un gruppo.
        \end{enumerate}
    }
\end{usecase}

\newpage

\begin{figure}[H] 
    \centering 
    \includegraphics[width=0.8\columnwidth]{usecase/sottocasi UC6.png} 
    \caption{UC6: Modifica di un gruppo}
\end{figure}

\begin{subusecase}{Modifica nome del gruppo}
    \label{sub:modifica-nome-gruppo}
    \usecaseactors{Utente autenticato}
    \usecasepre{L'utente ha aperto la \textit{web app}, è autenticato e sta modificando i dettagli di un gruppo}
    \usecasedesc{L'utente vuole modificare il nome di un gruppo}
    \usecasepost{L'utente ha modificato il nome di un gruppo}
    \usecasescenarioprincipale{
        \begin{enumerate}[nolistsep]
            \item L'utente modifica il nome di un gruppo.
        \end{enumerate}
    }
\end{subusecase}

\begin{subusecase}{Modifica descrizione del gruppo}
    \label{sub:modifica-descrizione-gruppo}
    \usecaseactors{Utente autenticato}
    \usecasepre{L'utente ha aperto la \textit{web app}, è autenticato e sta modificando i dettagli di un gruppo}
    \usecasedesc{L'utente vuole modificare la descrizione di un gruppo}
    \usecasepost{L'utente ha modificato la descrizione di un gruppo}
    \usecasescenarioprincipale{
        \begin{enumerate}[nolistsep]
            \item L'utente modifica la descrizione di un gruppo.
        \end{enumerate}
    }
\end{subusecase}

\begin{subusecase}{Modifica obbiettivi del gruppo}
    \label{sub:modifica-obbiettivi-gruppo}
    \usecaseactors{Utente autenticato}
    \usecasepre{L'utente ha aperto la \textit{web app}, è autenticato e sta modificando i dettagli di un gruppo}
    \usecasedesc{L'utente vuole modificare gli obbiettivi di un gruppo}
    \usecasepost{L'utente ha modificato gli obbiettivi di un gruppo}
    \usecasescenarioprincipale{
        \begin{enumerate}[nolistsep]
            \item L'utente modifica gli obbiettivi di un gruppo.
        \end{enumerate}
    }
\end{subusecase}


\begin{subusecase}{Modifica numero massimo di utenti}
    \label{sub:modifica-numero-utenti-gruppo}
    \usecaseactors{Utente autenticato}
    \usecasepre{L'utente ha aperto la \textit{web app}, è autenticato e sta modificando i dettagli di un gruppo}
    \usecasedesc{L'utente vuole modificare il numero massimo di utenti di un gruppo}
    \usecasepost{L'utente ha modificato il numero massimo di utenti di un gruppo}
    \usecasescenarioprincipale{
        \begin{enumerate}[nolistsep]
            \item L'utente modifica il numero massimo di utenti di un gruppo.
        \end{enumerate}
    }
\end{subusecase}

\begin{subusecase}{Modifica località del gruppo}
    \label{sub:modifica-località-gruppo}
    \usecaseactors{Utente autenticato}
    \usecasepre{L'utente ha aperto la \textit{web app}, è autenticato e sta modificando i dettagli di un gruppo}
    \usecasedesc{L'utente vuole modificare la località di un gruppo}
    \usecasepost{L'utente ha modificato la località di un gruppo}
    \usecasescenarioprincipale{
        \begin{enumerate}[nolistsep]
            \item L'utente modifica la località di un gruppo.
        \end{enumerate}
    }
\end{subusecase}
% TODO: se resta tempo implementare questa funzionalità
% \begin{subusecase}{Rimozione utente}
%     \label{sub:rimozione-utente}
%     \usecaseactors{Utente autenticato}
%     \usecasepre{L'utente ha aperto la \textit{web app}, è autenticato e sta modificando i dettagli di un gruppo}
%     \usecasedesc{L'utente vuole rimuovere un utente dal gruppo}
%     \usecasepost{L'utente ha rimosso un utente dal gruppo}
%     \usecasescenarioprincipale{
%         \begin{enumerate}[nolistsep]
%             \item L'utente visualizza la lista dei gruppi;
%             \item l'utente clicca su un gruppo sul che vuole modificare;
%             \item l'utente rimuove un utente dal gruppo.
%         \end{enumerate}
%     }
% \end{subusecase}

\begin{usecase}{Crea un nuovo gruppo}
    \label{uc:scenario-creazione-nuovo-gruppo}
    \usecaseactors{Utente autenticato}
    \usecasepre{L'utente ha aperto la \textit{web app} ed è autenticato}
    \usecasedesc{L'utente vuole creare un nuovo gruppo}
    \usecasepost{L'utente ha creato un nuovo gruppo}
    \usecasescenarioprincipale{
        \begin{enumerate}[nolistsep]
            \item L'utente crea un nuovo gruppo.
        \end{enumerate}
    }
\end{usecase}

\begin{figure}[H] 
    \centering 
    \includegraphics[width=1\columnwidth]{usecase/sottocasi UC7.png} 
    \caption{UC7: Crea un nuovo gruppo}
\end{figure}

\begin{subusecase}{Inserimento nome del gruppo}
    \label{sub:inserimento-nome-gruppo}
    \usecaseactors{Utente autenticato}
    \usecasepre{L'utente ha aperto la \textit{web app}, è autenticato e sta creando un nuovo gruppo}
    \usecasedesc{L'utente vuole inserire il nome del gruppo}
    \usecasepost{L'utente ha inserito il nome del gruppo}
    \usecasescenarioprincipale{
        \begin{enumerate}[nolistsep]
            \item L'utente inserisce il nome del gruppo.
        \end{enumerate}
    }
\end{subusecase}


\begin{subusecase}{Inserimento descrizione del gruppo}
    \label{sub:inserimento-descrizione-gruppo}
    \usecaseactors{Utente autenticato}
    \usecasepre{L'utente ha aperto la \textit{web app}, è autenticato e sta creando un nuovo gruppo}
    \usecasedesc{L'utente vuole inserire la descrizione del gruppo}
    \usecasepost{L'utente ha inserito la descrizione del gruppo}
    \usecasescenarioprincipale{
        \begin{enumerate}[nolistsep]
            \item L'utente inserisce la descrizione del gruppo.
        \end{enumerate}
    }
\end{subusecase}

\begin{subusecase}{Inserimento obbiettivi del gruppo}
    \label{sub:inserimento-obbiettivi-gruppo}
    \usecaseactors{Utente autenticato}
    \usecasepre{L'utente ha aperto la \textit{web app}, è autenticato e sta creando un nuovo gruppo}
    \usecasedesc{L'utente vuole inserire gli obbiettivi del gruppo}
    \usecasepost{L'utente ha inserito gli obbiettivi del gruppo}
    \usecasescenarioprincipale{
        \begin{enumerate}[nolistsep]
            \item L'utente inserisce gli obbiettivi del gruppo.
        \end{enumerate}
    }
\end{subusecase}


\begin{subusecase}{Inserimento numero massimo di utenti nel gruppo}
    \label{sub:inserimento-numero-utenti-gruppo}
    \usecaseactors{Utente autenticato}
    \usecasepre{L'utente ha aperto la \textit{web app}, è autenticato e sta creando un nuovo gruppo}
    \usecasedesc{L'utente vuole inserire il numero massimo di utenti del gruppo}
    \usecasepost{L'utente ha inserito il numero massimo di utenti del gruppo}
    \usecasescenarioprincipale{
        \begin{enumerate}[nolistsep]
            \item L'utente inserisce il numero massimo di utenti del gruppo.
        \end{enumerate}
    }
\end{subusecase}

\begin{subusecase}{Inserimento località del gruppo}
    \label{sub:inserimento-località-gruppo}
    \usecaseactors{Utente autenticato}
    \usecasepre{L'utente ha aperto la \textit{web app}, è autenticato e sta creando un nuovo gruppo}
    \usecasedesc{L'utente vuole inserire la località del gruppo}
    \usecasepost{L'utente ha inserito la località del gruppo}
    \usecasescenarioprincipale{
        \begin{enumerate}[nolistsep]
            \item L'utente inserisce la località del gruppo.
        \end{enumerate}
    }
\end{subusecase}

\begin{usecase}{Elimina un gruppo}
    \label{uc:scenario-elimina-gruppo}
    \usecaseactors{Utente autenticato}
    \usecasepre{L'utente ha aperto la \textit{web app}, è autenticato e si trova nella pagina di modifica di un gruppo}
    \usecasedesc{L'utente vuole eliminare un gruppo}
    \usecasepost{L'utente ha eliminato un gruppo}
    \usecasescenarioprincipale{
        \begin{enumerate}[nolistsep]
            \item L'utente elimina un gruppo.
        \end{enumerate}
    }
\end{usecase}


\begin{usecase}{Partecipa ad un gruppo}
    \label{uc:scenario-partecipa-gruppo}
    \usecaseactors{Utente autenticato}
    \usecasepre{L'utente ha aperto la \textit{web app}, è autenticato e si trova nella pagina dei dettagli di un gruppo a cui non partecipa}
    \usecasedesc{L'utente vuole partecipare ad un gruppo}
    \usecasepost{L'utente ha partecipato al gruppo}
    \usecasescenarioprincipale{
        \begin{enumerate}[nolistsep]
            \item L'utente partecipa al gruppo.
        \end{enumerate}
    }
\end{usecase}


% \begin{usecase}{Visualizzazione errore numero di utenti massimo raggiunto}
%     \label{uc:errore-numero-utenti}
%     \usecaseactors{Utente autenticato}
%     \usecasepre{L'utente ha aperto la \textit{web app} ed è autenticato}
%     \usecasedesc{L'utente vuole partecipare ad un gruppo}
%     \usecasepost{L'utente non si unisce al gruppo}
%     \usecasescenarioprincipale{
%         \begin{enumerate}[nolistsep]
%             \item l'utente sceglie di partecipare ad un gruppo;
%             \item viene visualizzato un messaggio di errore a causa della capienza massima raggiunta dal gruppo.
%         \end{enumerate}
%     }
% \end{usecase}


\begin{usecase}{Annulla partecipazione ad un gruppo}
    \label{uc:scenario-annulla-partecipazione}
    \usecaseactors{Utente autenticato}
    \usecasepre{L'utente ha aperto la \textit{web app}, è autenticato e si trova nella pagina dei dettagli di un gruppo a cui partecipa}
    \usecasedesc{L'utente vuole annullare la partecipazione al gruppo}
    \usecasepost{L'utente ha annullato la partecipazione al gruppo}
    \usecasescenarioprincipale{
        \begin{enumerate}[nolistsep]
            \item L'utente annulla la partecipazione al gruppo.
        \end{enumerate}
    }
\end{usecase}

\begin{usecase}{Visualizzazione lista delle will degli utenti appartenenti agli stessi gruppi}
    \label{uc:scenario-visualizzazione-will}
    \usecaseactors{Utente autenticato}
    \usecasepre{L'utente ha aperto la \textit{web app} ed è autenticato}
    \usecasedesc{L'utente vuole visualizzare la lista delle \textit{will} degli utenti appartenenti agli stessi gruppi}
    \usecasepost{L'utente ha visualizzato la lista delle \textit{will} degli utenti appartenenti agli stessi gruppi}
    \usecasescenarioprincipale{
        \begin{enumerate}[nolistsep]
            \item L'utente visualizza la lista delle \textit{will} degli utenti appartenenti agli stessi gruppi.
        \end{enumerate}
    }
\end{usecase}

\newpage 
 
\begin{figure}[H] 
    \centering 
    \includegraphics[width=1.2\columnwidth]{usecase/sottocasi UC11.png} 
    \caption{UC11: Vis. lista \textit{will} degli utenti appartenenti agli stessi gruppi}
\end{figure}

\begin{subusecase}{Visualizzazione singola will in lista}
    \label{sub:visualizzazione-singola-will}
    \usecaseactors{Utente autenticato}
    \usecasepre{L'utente ha aperto la \textit{web app}, è autenticato e sta visualizzando la lista delle \gls{will}}
    \usecasedesc{L'utente vuole visualizzare una singola \gls{will} dalla lista delle \gls{will}}
    \usecasepost{L'utente ha visualizzato una singola \gls{will} dalla lista delle \gls{will}}
    \usecasescenarioprincipale{
        \begin{enumerate}[nolistsep]
            \item L'utente visualizza una singola \gls{will} dalla lista delle \gls{will}.
        \end{enumerate}
    }
\end{subusecase}

\begin{figure}[H] 
    \centering 
    \includegraphics[width=1.2\columnwidth]{usecase/sottocasi UC11.1.png} 
    \caption{UC11.1: Vis. singola \textit{will}}
\end{figure}


\begin{subsubusecase}{Visualizzazione nome del proprietario}
    \label{subsub:visualizzazione-will-nome-proprietario}
    \usecaseactors{Utente autenticato}
    \usecasepre{L'utente ha aperto la \textit{web app}, è autenticato e sta visualizzando un \gls{will} dalla lista delle \gls{will}}
    \usecasedesc{L'utente vuole visualizzare il nome del proprietario della \gls{will} dalla lista delle \gls{will}}
    \usecasepost{L'utente ha visualizzato il nome del proprietario della \gls{will} dalla lista delle \gls{will}}
    \usecasescenarioprincipale{
        \begin{enumerate}[nolistsep]
            \item L'utente visualizza il nome del proprietario della \gls{will} dalla lista delle \gls{will}.
        \end{enumerate}
    }
\end{subsubusecase}

\begin{subsubusecase}{Visualizzazione descrizione}
    \label{subsub:visualizzazione-will-descrizione}
    \usecaseactors{Utente autenticato}
    \usecasepre{L'utente ha aperto la \textit{web app}, è autenticato e sta visualizzando un \gls{will} dalla lista delle \gls{will}}
    \usecasedesc{L'utente vuole visualizzare la descrizione della \gls{will} dalla lista delle \gls{will}}
    \usecasepost{L'utente ha visualizzato la descrizione della \gls{will} dalla lista delle \gls{will}}
    \usecasescenarioprincipale{
        \begin{enumerate}[nolistsep]
            \item L'utente visualizza la descrizione della \gls{will} dalla lista delle \gls{will}.
        \end{enumerate}
    }
\end{subsubusecase}

\begin{subsubusecase}{Visualizzazione data}
    \label{subsub:visualizzazione-will-data}
    \usecaseactors{Utente autenticato}
    \usecasepre{L'utente ha aperto la \textit{web app}, è autenticato e sta visualizzando un \gls{will} dalla lista delle \gls{will}}
    \usecasedesc{L'utente vuole visualizzare la data in cui verrà effettuata l'uscita}
    \usecasepost{L'utente ha visualizzato la data in cui verrà effettuata l'uscita}
    \usecasescenarioprincipale{
        \begin{enumerate}[nolistsep]
            \item L'utente visualizza la data in cui verrà effettuata l'uscita.
        \end{enumerate}
    }
\end{subsubusecase}


\section{Tracciamento dei requisiti}
Da un'attenta analisi dei requisiti e degli use case effettuata sul progetto è stata stilata la tabella che traccia i requisiti in rapporto agli use case.\\
Sono stati individuati diversi tipi di requisiti e si è quindi fatto utilizzo di un codice identificativo per distinguerli.\\
Il codice dei requisiti è così strutturato R(F/Q/V)(N/D/O) dove:
\begin{enumerate}[nolistsep]
	\item[R =] requisito
    \item[F =] funzionale
    \item[Q =] qualitativo
    \item[V =] di vincolo
    \item[N =] obbligatorio (necessario)
    \item[D =] desiderabile
    \item[Z =] opzionale
\end{enumerate}
Nelle tabelle \ref{tab:requisiti-funzionali}, \ref{tab:requisiti-qualitativi} e \ref{tab:requisiti-vincolo} sono riassunti i requisiti e il loro tracciamento con gli use case delineati in fase di analisi.
\begin{table}[H]%
\caption{Tabella del tracciamento dei requisti funzionali}
\label{tab:requisiti-funzionali}
\begin{tabularx}{\textwidth}{lXl}
\hline\hline
\textbf{Requisito} & \textbf{Descrizione} & \textbf{Use Case}\\
\hline
RFO-1     & Il sistema deve permettere di visualizzare la lista dei gruppi & \nameref{uc:scenario-visualizzazione-lista-gruppi} \\
\hline
RFO-1.1   & Il sistema deve permettere di visualizzare un singolo gruppo nella lista dei gruppi & \nameref{sub:visualizzazione-singolo-gruppo} \\
\hline
RFO-1.1.1   & Il sistema deve permettere di visualizzare il nome di un gruppo dalla lista dei gruppi & \nameref{subsub:visualizzazione-nome-gruppo} \\
\hline
RFO-1.1.2   & Il sistema deve permettere di visualizzare la descrizione di un gruppo dalla lista dei gruppi & \nameref{subsub:visualizzazione-descrizione-gruppo} \\
\hline
RFO-2     & Il sistema deve permettere di visualizzare la lista dei gruppi creati dall'utente & \nameref{uc:scenario-visualizzazione-lista-gruppi-creati}\\
\hline
RFO-3     & Il sistema deve permettere di visualizzare la lista dei gruppi  a cui partecipa l'utente & \nameref{uc:scenario-visualizzazione-lista-gruppi-partecipa}\\
\hline
RFO-4     & Il sistema deve permettere di visualizzare la lista dei gruppi creati da altri utenti & \nameref{uc:scenario-visualizzazione-lista-gruppi-altri}\\
\hline
RFO-5     & Il sistema deve permettere di visualizzare i dettagli di un gruppo & \nameref{uc:scenario-visualizzazione-dettaglio-gruppo}\\
\hline
RFO-5.1     & Il sistema deve permettere di visualizzare il nome di un gruppo & \nameref{sub:visualizzazione-nome-gruppo}\\
\hline
RFO-5.2     & Il sistema deve permettere di visualizzare l'immagine di un gruppo & \nameref{sub:visualizzazione-immagine-gruppo}\\
\hline

RFO-5.3     & Il sistema deve permettere di visualizzare la lista dei partecipanti ad un gruppo & \nameref{sub:visualizzazione-utenti-gruppo}\\
\hline

RFO-5.3.1     & Il sistema deve permettere di visualizzare un partecipante dalla lista dei partecipanti ad un gruppo & \nameref{subsub:visualizzazione-partecipanti-gruppo}\\
\hline

RFO-5.3.1.1     & Il sistema deve permettere di visualizzare nome e cognome di un partecipante dalla lista dei partecipanti ad un gruppo & \nameref{subsubsub:visualizzazione-singolo-partecipante-gruppo}\\
\hline

RFO-5.4     & Il sistema deve permettere di visualizzare la descrizione di un gruppo & \nameref{sub:visualizzazione-descrizione-gruppo}\\
\hline

RFO-5.5     & Il sistema deve permettere di visualizzare gli obbiettivi di un gruppo & \nameref{sub:visualizzazione-obbiettivi-gruppo}\\
\hline

RFO-5.6     & Il sistema deve permettere di visualizzare  località di un gruppo & \nameref{sub:visualizzazione-località-gruppo}\\
\hline

RFO-6     & Il sistema deve permettere di modificare i dettagli di un gruppo & \nameref{uc:scenario-modifica-gruppo}\\
\hline

RFO-6.1     & Il sistema deve permettere di modificare il nome di un gruppo & \nameref{sub:modifica-nome-gruppo}\\
\hline

RFO-6.2     & Il sistema deve permettere di modificare la descrizione di un gruppo & \nameref{sub:modifica-descrizione-gruppo}\\
\hline


RFO-6.3     & Il sistema deve permettere di modificare gli obbiettivi di un gruppo & \nameref{sub:modifica-obbiettivi-gruppo}\\
\hline


RFO-6.4     & Il sistema deve permettere di modificare il numero massimo di utenti di un gruppo & \nameref{sub:modifica-numero-utenti-gruppo}\\
\hline

RFO-6.5     & Il sistema deve permettere di modificare la località di un gruppo & \nameref{sub:modifica-località-gruppo}\\
\hline


RFO-7     & Il sistema deve permettere di creare un nuovo gruppo & \nameref{uc:scenario-creazione-nuovo-gruppo}\\
\hline

RFO-7.1     & Il sistema deve permettere di inserire il nome di un gruppo & \nameref{sub:inserimento-nome-gruppo}\\
\hline

RFO-7.2     & Il sistema deve permettere di inserire la descrizione di un gruppo & \nameref{sub:inserimento-descrizione-gruppo}\\
\hline


RFO-7.3     & Il sistema deve permettere di inserire gli obbiettivi di un gruppo & \nameref{sub:inserimento-obbiettivi-gruppo}\\
\hline

RFO-7.4     & Il sistema deve permettere di inserire il numero massimo di utenti del gruppo & \nameref{sub:inserimento-numero-utenti-gruppo}\\
\hline

RFO-7.5     & Il sistema deve permettere di inserire la località del gruppo & \nameref{sub:inserimento-località-gruppo}\\
\hline

RFO-8     & Il sistema deve permettere di eliminare un gruppo & \nameref{uc:scenario-elimina-gruppo}\\
\hline

RFO-9     & Il sistema deve permettere di partecipare ad un gruppo & \nameref{uc:scenario-partecipa-gruppo}\\
\hline

RFO-10     & Il sistema deve permettere di annullare la partecipazione & \nameref{uc:scenario-annulla-partecipazione}\\
\hline

RFO-11     & Il sistema deve permettere di visualizzare la lista delle \textit{will} degli utenti appartenenti agli stessi gruppi & \nameref{uc:scenario-visualizzazione-will}\\

RFO-11.1     & Il sistema deve permettere di visualizzare una singola \gls{will} dalla lista delle \gls{will} & \nameref{sub:visualizzazione-singola-will}\\
\hline

RFO-11.1.1     & Il sistema deve permettere di visualizzare il nome del proprietario della \gls{will} dalla lista delle \gls{will} & \nameref{subsub:visualizzazione-will-nome-proprietario}\\

\hline


RFO-11.1.2     & Il sistema deve permettere di visualizzare la descrizione delle \gls{will} dalla lista delle \gls{will} & \nameref{subsub:visualizzazione-will-descrizione}\\

\hline

\end{tabularx}
\end{table}%

\begin{table}[H]%
\caption{Tabella del tracciamento dei requisiti qualitativi}
\label{tab:requisiti-qualitativi}
\begin{tabularx}{\textwidth}{lXl}
\hline\hline
\textbf{Requisito} & \textbf{Descrizione} & \textbf{Use Case}\\
\hline
RQD-1    & Le prestazioni del simulatore hardware deve garantire la giusta esecuzione dei test e non la generazione di falsi negativi & - \\
\hline
\end{tabularx}
\end{table}%

\begin{table}[H]%
\caption{Tabella del tracciamento dei requisiti di vincolo}
\label{tab:requisiti-vincolo}
\begin{tabularx}{\textwidth}{lXl}
\hline\hline
\textbf{Requisito} & \textbf{Descrizione} & \textbf{Use Case}\\
\hline
RVO-1    & La libreria per l'esecuzione dei test automatici deve essere riutilizzabile & - \\
\hline
\end{tabularx}
\end{table}%