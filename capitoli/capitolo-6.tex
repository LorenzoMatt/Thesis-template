% !TEX encoding = UTF-8
% !TEX TS-program = pdflatex
% !TEX root = ../tesi.tex

%**************************************************************
\chapter{Conclusioni}
\label{cap:conclusioni}
%**************************************************************
%**************************************************************
\section{Raggiungimento degli obiettivi}
Sono stati raggiunti gli obbiettivi fissati dello \textit{stage}: infatti sono stati raggiunti con un anticipo tale che mi ha permesso di realizzare anche delle attività che non erano inizialmente pianificate.


\begin{center}
    {\rowcolors{2}{color1}{color2}
      \renewcommand{\arraystretch}{1}
      \begin{longtable}{
        |>{\centering\arraybackslash}p{60pt}
        |>{\centering\arraybackslash}p{220pt}
        |>{\centering\arraybackslash}p{60pt}|}
  
        \rowcolor{antimaincolor!0}
        \caption{\label{tab:raggiungimento-obbiettivi}Tabella raggiungimento degli obbiettivi}                                             \\
  
        \hline
        \rowcolor{maincolor}
        \color{antimaincolor}{Codice}                                                                 &
        \color{antimaincolor}{Descrizione}                                                               &
        \color{antimaincolor}{Esito}                                                                               \\
        \hline
        \endhead
  
        \rowcolor{maincolor}
        \color{antimaincolor}{Codice}                                                                 &
        \color{antimaincolor}{Descrizione}                                                               &
        \color{antimaincolor}{Esito}                                                                               \\
        \hline
        \endfoot
  
        O01     & Acquisizione competenze sulle tematiche sopra descritte & Soddisfatto \\
        \hline
        O02    & Capacità di raggiungere gli obiettivi richiesti in autonomia seguendo il cronoprogramma & Soddisfatto \\
        \hline
        O03     & Portare a termine l’implementazione dei \glspl{microservizio} richiesti con una percentuale di superamento pari al 80 & Soddisfatto \\
        \hline
        D01     & Portare a termine l’implementazione dei \glspl{microservizio} richiesti con una percentuale di superamento pari al 100 & Soddisfatto \\
        \hline
        D02     & Utilizzo della \gls{containerizzazione} per portare tutti i \glspl{microservizio} su Docker & Soddisfatto \\
        \hline
                
                                                                     
      \end{longtable}
      \renewcommand{\arraystretch}{1}
    }
  
  \end{center}

%**************************************************************
\section{Conoscenze acquisite}
Nel corso dello \textit{stage} ho appreso nuove tecnologie per la realizzazione della parte sia \textit{back end} sia \textit{front end}. Tra le prime spiccano l'utilizzo di Spring Boot e Spring Data JPA, oltre che i linguaggio Java. Ho, infatti, avuto modo di conoscere uno  dei \gls{framework}  più importanti e vasti per quanto concerne la programmazione in Java.\\
La conoscenza del linguaggio Java è sicuramente uno strumento molto utile per la gestione della persistenza dei dati in un \textit{database} relazionale. Combinando queste conoscenze con il \gls{framework} Spring è stato possibile creare applicazioni \textit{Restful} in maniera semplice e veloce.\\
Per quanto riguarda il lato \textit{front end}, Angular utilizzato assieme al linguaggio Typescript mi ha permesso di sviluppare, attraverso con le sue numerose funzionalità tra cui \textit{templating}, \textit{two-way binding}, modularizzazione, gestione \gls{API} \textit{Restful} e \textit{dependency injection}, in maniera efficiente ed efficace.  

%**************************************************************
\section{Valutazione personale}
Le aspettative sia dell’azienda che mie sono state ampiamente superate. Per questo motivo, ritengo i risultati ottenuti nel corso dello \textit{stage} sicuramente un successo, anche in virtù delle capacità acquisite. Quanto prodotto ha migliorato il \textit{software} già in possesso all'azienda, e può essere un punto di partenza per l'implementazione di nuove funzionalità. Basterà infatti prendere spunto da quanto integrato nel corso dello \textit{stage} per avere un'idea chiara su come implementare nuove funzionalità. \\
L'utilizzo di due dei \gls{framework} più importanti e popolari per lo sviluppo di applicazioni Java e per applicazioni \textit{web} dinamiche concorrono a consolidare le mie conoscenze in Java, ed aggiungere nel mio bagaglio di conoscenze un linguaggio a me nuovo in TypeScript.  \\
Infine, considero l'attività di \textit{stage} molto utile per capire come funziona un'azienda e come ci si relaziona con i colleghi. Questo vale, in particolare, per \myCompany: infatti, mi sono relazionato con lavoratori (e colleghi) che si occupano di sviluppo di applicazioni usufruendo dei \gls{framework} Spring e Angular.\\
Quanto vissuto in queste 8 settimane rappresenterà un bagaglio di esperienze prezioso per il mio futuro lavorativo e non.
