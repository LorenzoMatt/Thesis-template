\begin{center}
    {\rowcolors{2}{color1}{color2}
      \renewcommand{\arraystretch}{1.3}
      \begin{longtable}{
        |>{\centering\arraybackslash}p{60pt}
        |>{\centering\arraybackslash}p{220pt}
        |>{\centering\arraybackslash}p{60pt}|}
  
        \rowcolor{antimaincolor!0}
        \caption{\label{tab:requisiti-funzionali}Tabella del tracciamento dei requisiti funzionali}                                             \\
  
        \hline
        \rowcolor{maincolor}
        \color{antimaincolor}{Requisito}                                                                 &
        \color{antimaincolor}{Descrizione}                                                               &
        \color{antimaincolor}{Fonte}                                                                               \\
        \hline
        \endhead
  
        \rowcolor{maincolor}
        \color{antimaincolor}{Requisito}                                                                 &
        \color{antimaincolor}{Descrizione}                                                               &
        \color{antimaincolor}{Fonte}                                                                               \\
        \hline
        \endfoot
  
        RFO-1     & Il sistema deve permettere di visualizzare la lista dei gruppi & \nameref{uc:scenario-visualizzazione-lista-gruppi} \\
        \hline
        RFO-1.1   & Il sistema deve permettere di visualizzare un singolo gruppo nella lista dei gruppi & \nameref{sub:visualizzazione-singolo-gruppo} \\
        \hline
        RFO-1.1.1   & Il sistema deve permettere di visualizzare il nome di un gruppo dalla lista dei gruppi & \nameref{subsub:visualizzazione-nome-gruppo} \\
        \hline
        RFO-1.1.2   & Il sistema deve permettere di visualizzare la descrizione di un gruppo dalla lista dei gruppi & \nameref{subsub:visualizzazione-descrizione-gruppo} \\
        \hline
        RFO-2     & Il sistema deve permettere di visualizzare la lista dei gruppi creati dall'utente & \nameref{uc:scenario-visualizzazione-lista-gruppi-creati}\\
        \hline
        RFO-3     & Il sistema deve permettere di visualizzare la lista dei gruppi  a cui partecipa l'utente & \nameref{uc:scenario-visualizzazione-lista-gruppi-partecipa}\\
        \hline
        RFO-4     & Il sistema deve permettere di visualizzare la lista dei gruppi creati da altri utenti & \nameref{uc:scenario-visualizzazione-lista-gruppi-altri}\\
        \hline
        RFO-5     & Il sistema deve permettere di visualizzare i dettagli di un gruppo & \nameref{uc:scenario-visualizzazione-dettaglio-gruppo}\\
        \hline
        RFO-5.1     & Il sistema deve permettere di visualizzare il nome di un gruppo & \nameref{sub:visualizzazione-nome-gruppo}\\
        \hline
        RFO-5.2     & Il sistema deve permettere di visualizzare l'immagine di un gruppo & \nameref{sub:visualizzazione-immagine-gruppo}\\
        \hline
        
        RFO-5.3     & Il sistema deve permettere di visualizzare la lista dei partecipanti ad un gruppo & \nameref{sub:visualizzazione-utenti-gruppo}\\
        \hline
        
        RFO-5.3.1     & Il sistema deve permettere di visualizzare un partecipante dalla lista dei partecipanti ad un gruppo & \nameref{subsub:visualizzazione-partecipanti-gruppo}\\
        \hline
        
        RFO-5.3.1.1     & Il sistema deve permettere di visualizzare nome e cognome di un partecipante dalla lista dei partecipanti ad un gruppo & \nameref{subsubsub:visualizzazione-singolo-partecipante-gruppo}\\
        \hline
        
        RFO-5.4     & Il sistema deve permettere di visualizzare la descrizione di un gruppo & \nameref{sub:visualizzazione-descrizione-gruppo}\\
        \hline
        
        RFO-5.5     & Il sistema deve permettere di visualizzare gli obbiettivi di un gruppo & \nameref{sub:visualizzazione-obbiettivi-gruppo}\\
        \hline
        
        RFO-5.6     & Il sistema deve permettere di visualizzare  località di un gruppo & \nameref{sub:visualizzazione-località-gruppo}\\
        \hline
        
        RFO-6     & Il sistema deve permettere di modificare i dettagli di un gruppo & \nameref{uc:scenario-modifica-gruppo}\\
        \hline
        
        RFO-6.1     & Il sistema deve permettere di modificare il nome di un gruppo & \nameref{sub:modifica-nome-gruppo}\\
        \hline
        
        RFO-6.2     & Il sistema deve permettere di modificare la descrizione di un gruppo & \nameref{sub:modifica-descrizione-gruppo}\\
        \hline
        
        
        RFO-6.3     & Il sistema deve permettere di modificare gli obbiettivi di un gruppo & \nameref{sub:modifica-obbiettivi-gruppo}\\
        \hline
        
        
        RFO-6.4     & Il sistema deve permettere di modificare il numero massimo di utenti di un gruppo & \nameref{sub:modifica-numero-utenti-gruppo}\\
        \hline
        
        RFO-6.5     & Il sistema deve permettere di modificare la località di un gruppo & \nameref{sub:modifica-località-gruppo}\\
        \hline
        
        
        RFO-7     & Il sistema deve permettere di creare un nuovo gruppo & \nameref{uc:scenario-creazione-nuovo-gruppo}\\
        \hline
        
        RFO-7.1     & Il sistema deve permettere di inserire il nome di un gruppo & \nameref{sub:inserimento-nome-gruppo}\\
        \hline
        
        RFO-7.2     & Il sistema deve permettere di inserire la descrizione di un gruppo & \nameref{sub:inserimento-descrizione-gruppo}\\
        \hline
        
        
        RFO-7.3     & Il sistema deve permettere di inserire gli obbiettivi di un gruppo & \nameref{sub:inserimento-obbiettivi-gruppo}\\
        \hline
        
        RFO-7.4     & Il sistema deve permettere di inserire il numero massimo di utenti del gruppo & \nameref{sub:inserimento-numero-utenti-gruppo}\\
        \hline
        
        RFO-7.5     & Il sistema deve permettere di inserire la località del gruppo & \nameref{sub:inserimento-località-gruppo}\\
        \hline
        
        RFO-8     & Il sistema deve permettere di eliminare un gruppo & \nameref{uc:scenario-elimina-gruppo}\\
        \hline
        
        RFO-9     & Il sistema deve permettere di partecipare ad un gruppo & \nameref{uc:scenario-partecipa-gruppo}\\
        \hline
        
        RFO-10     & Il sistema deve permettere di annullare la partecipazione & \nameref{uc:scenario-annulla-partecipazione}\\
        \hline
        
        RFO-11     & Il sistema deve permettere di visualizzare la lista delle \textit{will} degli utenti appartenenti agli stessi gruppi & \nameref{uc:scenario-visualizzazione-will}\\
        
        RFO-11.1     & Il sistema deve permettere di visualizzare una singola \gls{will} dalla lista delle \gls{will} & \nameref{sub:visualizzazione-singola-will}\\
        \hline
        
        RFO-11.1.1     & Il sistema deve permettere di visualizzare il nome del proprietario della \gls{will} dalla lista delle \gls{will} & \nameref{subsub:visualizzazione-will-nome-proprietario}\\
        
        \hline
        
        
        RFO-11.1.2     & Il sistema deve permettere di visualizzare la descrizione delle \gls{will} dalla lista delle \gls{will} & \nameref{subsub:visualizzazione-will-descrizione}\\
        
        \hline                                                           
      \end{longtable}
      \renewcommand{\arraystretch}{1}
    }
  
  \end{center}