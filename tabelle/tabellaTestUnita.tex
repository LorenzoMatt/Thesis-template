\begin{center}
    {\rowcolors{2}{color1}{color2}          % rows height
      \renewcommand{\arraystretch}{1.60}
      \begin{longtable}{
        |>{\centering\arraybackslash}p{48pt}
        |>{\centering\arraybackslash}p{308pt}
        |>{\centering\arraybackslash}p{27pt}|}
  
  
        \rowcolor{antimaincolor!0}
        \caption{\label{tab:Test di unita}Test di unità}                                                                                                             \\
  
        \hline                                    % header
        \rowcolor{maincolor}                      % header color
        \color{antimaincolor}{Codice}      &
        \color{antimaincolor}{Descrizione} &
        
        \color{antimaincolor}{Esito}                                                                                                                                 \\
        \hline
        \endhead
  
        \hline                                    % header
        \rowcolor{maincolor}                      % header color
        \color{antimaincolor}{Codice}      &
        \color{antimaincolor}{Descrizione} &
        
        \color{antimaincolor}{Esito}                                                                                                                                 \\
        \hline
        \endfoot
  
        TU1                                & Il metodo \texttt{getAllGruppi} deve restituire tutti i gruppi presenti                                           & S \\
        TU2                                & Il metodo \texttt{getAllGruppi} deve restituire un \texttt{HttpStatus.NO\_CONTENT} nel caso non siano presenti gruppi                                      & S \\
        TU3                                & il metodo \texttt{getGruppo} deve restituire il gruppo specificato nel caso sia presente & S \\
        TU4                                & il metodo \texttt{getGruppo} deve restituire un \texttt{HttpStatus.NO\_CONTENT} nel caso non sia presente il gruppo & S \\
        TU5                                & Il metodo \texttt{getUtentiByGruppo} deve restituire gli utenti di un gruppo & S \\
        TU6                                & il metodo \texttt{createGruppo} deve creare un gruppo & S \\
        TU7                                & Il metodo \texttt{createGruppo} deve restituire un \texttt{ HttpStatus.BAD\_REQUEST} nel caso non esista il proprietario del gruppo specificato  & S \\
        TU8                                & Il metodo \texttt{createGruppo} deve restituire un \texttt{ HttpStatus.BAD\_REQUEST} nel caso che le coordinate geografiche inserite non siano valide & S \\
        TU9                                & Il metodo \texttt{modifyGruppo} deve modificare un gruppo & S \\
        TU10                                & Il metodo \texttt{modifyGruppo} deve restituire un \texttt{ HttpStatus.NOT\_MODIFIED} nel caso il gruppo specificato non esiste & S \\
        TU11                                & Il metodo \texttt{modifyGruppo} deve restituire un \texttt{ HttpStatus.BAD\_REQUEST} nel caso non esista il proprietario del gruppo specificato & S \\
        TU12                                & Il metodo \texttt{deleteGruppo} deve eliminare un gruppo & S \\
        TU13                                & Il metodo \texttt{deleteGruppo} deve deve restituire un \texttt{ HttpStatus.BAD\_REQUEST} nel caso non esista il gruppo specificato   & S \\
        TU14                                & Il metodo \texttt{getUsciteByGruppo} deve restituire le uscite di un gruppo & S \\
        TU15                                & Il metodo \texttt{getUsciteByGruppo} deve restituire un \texttt{ HttpStatus.BAD\_REQUEST} nel caso non esista il gruppo specificato   & S \\
        TU16                                & Il metodo \texttt{getGruppiOfUtente} deve restituire i gruppi a cui partecipa un utente   & S \\
        TU17                                & Il metodo \texttt{addUtenteToGruppo} deve aggiungere un utente ad un gruppo   & S \\
        TU18                                & Il metodo \texttt{addUtenteToGruppo} non deve aggiungere un nuovo utente nel caso il gruppo abbia superato il numero massimo di utenti    & S \\
        TU19                                & Il metodo \texttt{addUtenteToGruppo} deve restituire un \texttt{ HttpStatus.BAD\_REQUEST} nel caso non esista il gruppo specificato     & S \\
        TU20                                & Il metodo \texttt{removeUtenteFromGruppo} deve rimuovere un utente da un gruppo    & S \\
        TU21                                & Il metodo \texttt{removeUtenteFromGruppo} deve restituire un \texttt{ HttpStatus.BAD\_REQUEST} nel caso non esista il gruppo specificato   & S \\
        TU22                                & Il metodo \texttt{deleteUtenteFromJointUtenteCrew} deve rimuovere un utente dalla tabella \texttt{joint\_utenti\_crew}    & S \\
        TU23                                & Il metodo \texttt{addUscitaToGruppoCrew} deve aggiungere un'uscita ad un gruppo    & S \\
        TU24                                & Il metodo \texttt{addUscitaToGruppoCrew}  deve restituire un \texttt{HttpStatus.BAD\_REQUEST} nel caso non esista l'uscita specificata    & S \\
        TU25                                & Il metodo \texttt{addUscitaToGruppoCrew} deve restituire un \texttt{HttpStatus.BAD\_REQUEST} nel caso non esista il gruppo specificato    & S \\
        TU26                                & Il metodo \texttt{removeUscitaFromGruppo} deve rimuovere un'uscita dal gruppo specificato    & S \\
        TU27               & Il metodo \texttt{removeUscitaFromGruppo} deve restituire un \texttt{HttpStatus.BAD\_REQUEST} nel caso non esista l'uscita specificata    & S \\
        TU28               & Il metodo \texttt{removeUscitaFromGruppo}deve restituire un \texttt{HttpStatus.BAD\_REQUEST} nel caso non esista il gruppo specificato    & S \\
        TU29               & Il metodo \texttt{deleteUscitaFromJointUscitaCrew}deve rimuovere un'uscita dal gruppo specificato    & S \\
        TU30               & Il metodo \texttt{deleteUscitaFromJointUscitaCrew}deve restituire un \texttt{HttpStatus.BAD\_REQUEST} nel caso non esista l'uscita specificata   & S \\
        TU29               & Il metodo \texttt{deleteUscitaFromJointUscitaCrew}deve restituire un \texttt{HttpStatus.BAD\_REQUEST} nel caso non esista il gruppo specificato    & S \\
        
        
        
        
      \end{longtable}
    }
  \end{center}