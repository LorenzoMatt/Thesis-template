
% Tabella da personalizzare in base alle ore delle attività

\begin{tabularx}{\textwidth}{|c|X|}
	\hline
	\textbf{Durata in ore} & \textbf{Descrizione dell'attività} \\\hline
	
	\textbf{160} & \textbf{Formazione sulle tecnologie} \\ \hdashline
 
    \multirow{7}{0cm}\\ 
    \textit{18} & 
    \textit{Studio Java Standard Edition e tool di sviluppo} \\
    \textit{18} & 
    \textit{Studio architettura a \glspl{microservizio}} \\
    \textit{4} & 
    \textit{Ripasso dei principi della buona programmazione (SOLID, CleanCode)} \\


    \textit{10} & 
    \textit{Studio teorico dell’architettura a \glspl{microservizio}: passaggio da monolite ad architetture a \glspl{microservizio}} \\
    \textit{15} & 
    \textit{Studio teorico dell’architettura a \glspl{microservizio}: Api Gateway, Service Discovery e Service Registry, Circuit Breaker e Saga Pattern} \\
    \textit{15} & 
    \textit{Studio Spring Core/Spring Boot} \\ 

    
    \textit{20} & 
    \textit{Studio servizi REST e framework Spring Data REST} \\

    \textit{60} &
    \textit{Studio ORM, in particolare il framework Spring Data JPA} \\

    \hline
    
    \textbf{40} & \textbf{Definizione architettura di riferimento e relativa documentazione} \\ \hdashline
 
    \multirow{3}{0cm}\\ 
    \textit{14} & 
    \textit{Analisi del problema e del dominio applicativo} \\
    \textit{22} & 
    \textit{Progettazione della piattaforma e relativi test} \\
    \textit{4} & 
    \textit{Stesura documentazione relativa ad analisi e progettazione} \\
    \hline
    \textbf{80} & \textbf{Implementazione del nuovo servizio} \\ 
    \hline
    
    \textbf{40} & \textbf{Collaudo Finale}  \\ \hdashline
 
    \multirow{4}{0cm}\\ 
    \textit{30} & 
    \textit{Collaudo} \\
    \textit{6} & 
    \textit{Stesura documentazione finale} \\
    \textit{2} & 
    \textit{Incontro di presentazione della piattaforma con gli stakeholders} \\
    \textit{2} & 
    \textit{Live demo di tutto il lavoro di stage} \\
    \hline
	
	\textbf{Totale ore} & \multicolumn{1}{|c|}{\textbf{320}} \\\hline
	
	
\end{tabularx}